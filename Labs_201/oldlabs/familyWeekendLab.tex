% book.tex -- Copyright (c) 2004, Brian C. Ladd
%
%  Author : Brian C. Ladd 
%  Created: Wed Jul 07 12:03:49 2004 by blad
%  Revised: Fri Dec 28 15:27:43 2007 by blad
%  
\def\FileCreated{Wed Jul 07 12:03:49 2004}
\def\FileRevised{Fri Dec 28 15:27:43 2007}
\documentclass[12pt,twoside]{memoir}
\usepackage{epsfig}
\usepackage{graphicx}
\usepackage[usenames]{color}
\usepackage{xltxtra}
\usepackage{fontspec}
\usepackage{textcomp}
\usepackage{pstricks}
\usepackage{listings}[2000/08/23] 

\usepackage{rotating}
\usepackage{exercise}
% ----- Fonts -----
\defaultfontfeatures{Scale=MatchLowercase,Mapping=tex-text}
\setmainfont{Gentium Book Basic}
\setmonofont{Bitstream Vera Sans Mono}
\setsansfont{Arial}
% ----- Fonts -----

\definecolor{nicered}{rgb}{.647,.129,.149}
\definecolor{listinggray}{gray}{0.1}
\definecolor{templategrey}{gray}{0.85}
\definecolor{commandlinebackground}{gray}{.90}
\definecolor{commandlineforeground}{gray}{0.20,}
\definecolor{commandpromptforeground}{gray}{0.55}

\newcommand\code[1]{\lstinline^#1^}
\newcommand\fname[1]{\lstinline^#1^}
\newcommand\note[1]{\unskip\footnote{#1}}
\newcommand\foreign[1]{\emph{#1}}
\newcommand\ensurecomma{%
  \@ifnextchar,{}{\@latex@error{Don’t forget the comma!}{}}}
\newcommand\eg{\foreign{e.g.}\ensurecomma}
\newcommand\ie{\foreign{i.e.}\ensurecomma}
\newcommand\cf{\foreign{cf.\@}}

\newcommand\ensuresingleperiod{\@ifnextchar.{}{.\@}}
\newcommand\etc{\foreign{etc}\ensuresingleperiod}
\newcommand\etal{\foreign{et al}\ensuresingleperiod}

\DeclareGraphicsExtensions{.eps,.pdf,.png,.gif,.jpg}

\newcounter{ProgrammingProblem}
\setcounter{ProgrammingProblem}{0}%

\newenvironment{LabExercises}{%
\renewcommand{\ExerciseListName}{Question}%
\renewcommand{\ExerciseListHeader}{\textbf{%
   Checkpoint\ExerciseHeaderNB. }}
\begin{ExerciseList}}%
{\end{ExerciseList}}
\newcommand{\LabExercise}{\Exercise[name={Lab Phase\ExerciseHeaderNB},counter={ProgrammingProblem}]}


\lstset{language=java,
  basicstyle=\small\ttfamily,
  numbers=left, 
  numberstyle=\tiny, 
  stepnumber=1, 
  numbersep=5pt,
  frame=single,
  captionpos=b,
  rulecolor=\color{nicered}
}

\lstdefinelanguage{cline}
{
  morecomment=[s][\color{commandpromptforeground}]{ }{ },
} 

\lstnewenvironment{commandline}[1][]
  {\lstset{language=cline,numbers=none,frame=none,backgroundcolor=\color{commandlinebackground},basicstyle=\color{commandlineforeground},nolol,#1}}
  {}

\setlength{\hoffset}{0in}
\setlength{\voffset}{0in}
\settypeblocksize{9.5in}{7.5in}{*}
\setlrmargins{0.5in}{*}{1}
\setulmargins{0.75in}{*}{*}
\setheadfoot{\onelineskip}{2\onelineskip}
\checkandfixthelayout

\begin{document}

\title{Fall 2008 CS1 Family Weekend  Lab}
\maketitle

\begin{itemize}
\item Learn how to start (and stop) the Emacs text editor
\item Learn how to create a \texttt{.java} file
\item Learn how to compile a \texttt{.java} file
\item Learn how to run a \texttt{.java} (or rather, \texttt{.class}) file
\item Learn how to correct simple \emph{compiler errors}
\end{itemize}

\begin{LabExercises}
\LabExercise Launch a terminal.

Linux provides a desktop environment similar to that found in other
operating systems (Windows XP/Vista or Mac OS X). It also offers a
\emph{command-line interface}, or a \emph{shell} where the user can
type commands into a command interpreter which runs the commands and
then prompts the user for a new command. In this class we will be
using the command-line interface a great deal.

A shell runs inside of a terminal; go to the menu at the upper-left of
the desktop and find \textbf{Accessories $|$ Terminal}. Launching it
should give you a window that has a prompt something like this:

\begin{commandline}
 laddbc@admin-156-97:~$ _
\end{commandline}
% $

The bit before the \code{@} is your login name (Dr. Ladd's is
\code{laddbc}). Between the \code{@} and the \code{:} is the name of
the machine you are logged into. Between the \code{:} and the \code{\$}
is the \emph{current directory path}. The \code{\~} is Linux shorthand
for your home directory. 

Note that the \code{_} after the prompt represents where you can
type. It will not be shown in other examples which will show commands
typed in after the prompts.  

The Linux file system uses hierarchical \emph{directories} (sometimes
called \emph{folders}), containers for files and other
directories. You will now change to the \fname{Lab1} directory using
the \code{cd} (change directory) command:

\begin{commandline}
 laddbc@admin-156-97:~$ cd Lab1
 laddbc@admin-156-97:~/Lab1$ ls -a
 . .. NewtonsApple.java
 laddbc@admin-156-97:~/Lab1$ 
\end{commandline}
%$
\noindent
The current directory, displayed in the command prompt, changes as you
navigate into directories below your home directory. The \code{ls -a}
command tells the shell to list all files and directories in the
current directory.  \fname{.} and \fname{..} are special names for the
current and parent directories. \fname{NewtonsApple.java} is the Java
program template you will extend in this lab. 

How would you change the current directory back to \code{\~}?

\LabExercise Starting and stopping \fname{emacs}

In CS1 we use a \emph{text editor} to enter our Java code; a text
editor is like a word processor except that the text editor does not
save formatting information. Instead it saves just the
characters. This is important because the Java compiler does not
understand formatting.

To run the \code{emacs} text editor, you can either type the name of
the program into the shell. When you type the name in, you
can also specify the name of a file you wish to edit. We can type:

\begin{commandline}
 laddbc@admin-156-97:~/CS1/Lab1$ emacs NewtonsApple.java &
 [1] 1234
 laddbc@admin-156-97:~/CS1/Lab1$ 
\end{commandline}

\noindent
What is that \code{&} doing there? It runs the command in the
background so we can keep using the shell. The following line is
information about the background process.

\fname{emacs} has a fairly standard menu bar. Under \textbf{File}
there are commands to open files, save files, and exit
\fname{emacs}. The keyboard shortcuts are also listed (\textbf{C-} in
front of a character means press the \textbf{Ctrl} key while pressing
the key).

The \textbf{JDE} menu is the \emph{Java Development Environment}. That
is where the commands to compile and run Java programs can be found.

Look at the code. The colors and indentation are provided to make it
easier for you to read. In Java, many things are held in \emph{blocks}
or containers between curly braces (\code{\{} \code{\}}). To emphasise
this relationship, lines end with opening curly braces and lines
contained within that block are indented. 

The lines that appear purple and have \code{//} or \code{/*} and
\code{*/} around them are comments. They are written by programmers
for other programmers. Java treats them as empty lines when compiling.

Enough preliminaries! It is time to compile a program.

Select \textbf{JDE $|$ Compile}. The blank line at the bottom of the
\code{emacs} window (known as the \emph{minibuffer} in \code{emacs}
parlance) reports something about starting the ``Beanshell'', then
the window splits horizontally and a window labeled \emph{*JDEE Compile
  Server*} appears with the output of the compilation. If all went
well, the output will look something like this:

\begin{lstlisting}[numbers=none]
CompileServer output:

-classpath /home/faculty/laddbc/CS1/Lab1:/home/faculty/laddbc/jar/fang.jar 
/home/faculty/laddbc/CS1/Lab1/NewtonsApple.java

Compilation finished at Sat Aug 23 16:37:57
\end{lstlisting}

To get rid of the split in the window you can, with the cursor in the
top window, select \textbf{File $|$ Unsplit Window} (or use the
keyboard shortcut of \textbf{C-x 1} to get just one (1) window). 

\LabExercise Run the program

If your program compiled without errors, select \textbf{JDE $|$ Run
  App}. The window splits again with Java output in the second pane
and a new window appears labeled \emph{NewtonsApple}. This is the
game. The buttons along the bottom row are provided by FANG for all
games.

\LabExercise Continue the program.

Update your program file by finding the comments given in each of the
examples below and entering the code shown just below them. Most of the
comments (and code) in this section belong in the \code{setup} method
though two new methods are defined at the bottom of the class.
\newpage
\begin{itemize}
\item \textbf{dropApple} We must create a new command for our
  game. The command, called \code{dropApple}:
\begin{lstlisting}[numbers=none]
    /**
     * New command to drop a new apple. Positions apple at top of
     * screen and increments the number of apples that have been
     * dropped.
     */
    public void dropApple() {
	apple.setLocation(randomDouble(), 0.0);
	applesDropped = applesDropped + 1;
    }
\end{lstlisting}

Go ahead and try compiling; the compiler is like a spell checker that
you should run early and often! You should probably save your program
and compile after each of these items.

\item \textbf{apple} The apple is an \code{OvalSprite} with equal
  diameters, making it a circle. It is colored red and then dropped
  (using our new command from above). All sprites must be added to the
  game.
\begin{lstlisting}[numbers=none]
	// initialize the apple: OvalSprite constructor takes an
	// x-diameter and a y-diameter (to make a circle, make them
	// the same). They are expressed in screens. getColor can find
	// named colors (here "red", "green", and "white" are used)
	apple = new OvalSprite(0.10, 0.10);
	apple.setColor(getColor("red"));
	dropApple();
	addSprite(apple);
\end{lstlisting}

\item \textbf{newton} A square (same dimension both ways) the same
  size as the apple. Colored green, positioned at the bottom of the
  screen, and added to the game.
\begin{lstlisting}[numbers=none]    
	// newton is small, green, and begins at the middle bottom of the
	// screen.
	newton = new RectangleSprite(0.1, 0.1);
	newton.setColor(getColor("green"));
	newton.setLocation(0.5, 0.9);
	addSprite(newton);
\end{lstlisting}

\item \textbf{updateScore} A new command that puts the number of
  apples dropped and the number of apples caught into the sprite on
  the screen and adjusts its position to keep it in the upper-left corner.
\begin{lstlisting}[numbers=none]    
    /**
     * Update displayScore sprite with current score (apples caught /
     * apples dropped).  Adjust position to make sure it is all
     * visible.
     */
    public void updateScore() {
	displayScore.setText("Score: " + applesCaught + "/" + applesDropped);
	displayScore.setLocation(displayScore.getWidth() / 2, 
				              displayScore.getHeight() / 2);
    }
\end{lstlisting}

\item \textbf{score counters} Initialize the number of apples dropped
  and caught to \code{0}; note that this is before calling either of
  the new commands (both of which use the score variables).
\begin{lstlisting}[numbers=none]    
	// initialize the score
	applesCaught = 0;
	applesDropped = 0;
\end{lstlisting}

\item \textbf{displayScore} Create the on-screen score display,
  set its color, and use \code{updateScore} to update its contents.
\begin{lstlisting}[numbers=none]    
	// The score is in the upper left corner of the screen in white
	displayScore = new StringSprite(0.10);
	displayScore.setColor(getColor("white"));
	updateScore(); 
	addSprite(displayScore);
\end{lstlisting}
\end{itemize}

Now, after you compile the last bit, you can again run the program
(and get something reasonable). The apple appears partially off the
top of the screen, the score in the upper-left corner, and the Newton
at the bottom center of the screen. 

The game is not yet playable; if you press \textbf{Start} you will
find that nothing happens. This is because FANG calls the
\code{advance} method as part of the main game loop and we have
nothing but comments in that method.

\textbf{Correcting Typos}: If you have any compiler errors,
\code{emacs} will show them to you. Consider, for example, misspelling
\code{Color} as \code{Colour}. Then the CompileServer output (in the
bottom window) would look like this:

\begin{lstlisting}[numbers=none]
CompileServer output:

-classpath /home/faculty/laddbc/CS1/Lab1:/home/faculty/laddbc/jar/fang.jar 
/home/faculty/laddbc/CS1/Lab1/NewtonsApple.java
/home/faculty/laddbc/CS1/Lab1/NewtonsApple.java:11: cannot find symbol
symbol  : method setColour(java.awt.Color)
location: class fang.sprites.OvalSprite
    apple.setColour(getColor("red"));
         ^
/home/faculty/laddbc/CS1/Lab1/NewtonsApple.java:15: cannot find symbol
symbol  : method setColour(java.awt.Color)
location: class fang.sprites.RectangleSprite
    newton.setColour(getColor("green"));
          ^
2 errors

Compilation exited abnormally with code 1 at Mon Aug 25 15:48:00  
\end{lstlisting}

\code{emacs} will move the cursor down to the first
\code{/home/faculty/...} line, and also move the cursor in the
\fname{NewtonsApple.java} buffer to line 11 (the line with the
error). The first line in the error window identifies the file and
line with the error and the error the Java compiler encountered. In
this case the error is \code{cannot find symbol}. The next line gives
the symbol that could not be found (\code{setColour}) and the line
after that gives the class where the Java compiler expected to find
the symbol (\code{OvalSprite}).

The next two lines are the offending line from the \fname{.java} file
and a line with a \lstinline{^} pointing to the spot where the Java
compiler realized there was an error. In this case, we see that the
method name in our listing was \code{setColor} though we typed
\code{setColour}. Removing the ``u'' fixes the problem.

After correcting one error you can move to the next error by pressing
\code{C-x `} (that is a back-tick, next to the \code{1} on most
keyboards).  The error message window and the source code window will
both move to the next error.
\newpage
\LabExercise Finish \texttt{NewtonsApple}

To finish \code{NewtonsApple}, we just find the appropriate comments
in \code{advance} and add the code below to move newton, move the
apple, and check if the apple has been caught or hit the ground.

\begin{itemize}
\item \textbf{Move newton} Set \code{newton} horizontal location to
  match the player's mouse's location.
\begin{lstlisting}[numbers=none]    
	// (1) Move newton so his x-coordinate matches x-coordinate of mouse
	Location2D position = getPlayer().getMouse().getLocation();
	if (position != null) {
	    newton.setX(position.x);
	}
\end{lstlisting}

\item \textbf{Move apple} Move \code{apple} down according to its falling
  velocity (in screens).
\begin{lstlisting}[numbers=none]    
	// (2) Translate apple down screen; velocity * time = distance
	apple.translateY(0.33 * secondsSinceLastCall);
\end{lstlisting}

\item \textbf{Catch?} If \code{newton} gets the \code{apple}, update
  score and drop a new apple.
\begin{lstlisting}[numbers=none]    
	// (3) Check if newton has "caught" the apple
	if (apple.intersects(newton)) {
	    applesCaught = applesCaught + 1; 
	    updateScore();
	    dropApple();
	}
\end{lstlisting}

\item \textbf{Splat?} If \code{apple} hits the bottom of the screen,
  drop a new apple.
\begin{lstlisting}[numbers=none]    
	// (4) Check if the apple has hit the ground (y-coordinate 1.0)
	if (apple.getY() >= 1.0) {
	    updateScore();
	    dropApple();
	}
\end{lstlisting}
\end{itemize}

\LabExercise Save, compile, and run \texttt{NewtonsApple}.

Save your work, compile it, correct any compilation errors, and run
it. You should have a working game running.  Normally we would now
upload the program to the on-line course management system.

\LabExercise Make \texttt{NewtonsPear} (If you have time)

Open a new file (in the same directory) called
\texttt{NewtonsPear.java}. Newton's pear is just like Newton's apple
but the pear is yellow rather than red. 

\begin{itemize}
\item Use \textbf{File $|$ Insert File...} in the new buffer. Select
  \texttt{NewtonsApple.java} to insert.
\item Save the file.
\item The file must have the name of the class updated. The name of
the class \emph{must} match the name of the file (less the
\fname{.java}). You can go to \textbf{Edit $|$ Search $|$ Replace}and
replace \code{NewtonsApple} with \code{NewtonsPear}.  
\item Similarly, the variable \code{apple} should be renamed
  \code{pear}. Use the same command to replace one with the other.
\item Finally, lets fix up line 16 so that instead of setting the
  color to ``red'', the color of the oval is set to ``yellow''.
\end{itemize}

Now save and compile your second program. It should look familiar but
with a small difference.

\end{LabExercises}


\end{document}

%%% Local Variables: 
%%% mode: latex
%%% TeX-master: t
%%% End: 

% LocalWords:  Moodle Ladd's login
