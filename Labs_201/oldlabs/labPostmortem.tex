\documentclass[10pt,oneside]{memoir}
\usepackage{epsfig}
\usepackage{graphicx}
\usepackage[usenames]{color}
\usepackage{xltxtra}
\usepackage{fontspec}
\usepackage{textcomp}
\usepackage{pstricks}
\usepackage{listings}[2000/08/23] 

\usepackage{rotating}
\usepackage{exercise}
% ----- Fonts -----
\defaultfontfeatures{Scale=MatchLowercase,Mapping=tex-text}
\setmainfont{Gentium Book Basic}
\setmonofont{Bitstream Vera Sans Mono}
\setsansfont{Arial}
% ----- Fonts -----

\definecolor{nicered}{rgb}{.647,.129,.149}
\definecolor{listinggray}{gray}{0.1}
\definecolor{templategrey}{gray}{0.85}
\definecolor{NewtonsApple}{gray}{.95}
\definecolor{commandlinebackground}{gray}{.90}
\definecolor{commandlineforeground}{gray}{0.20,}
\definecolor{commandpromptforeground}{gray}{0.55}

\newcommand\code[1]{\lstinline^#1^}
\newcommand\fname[1]{\texttt{#1}}
\newcommand\note[1]{\unskip\footnote{#1}}
\newcommand\foreign[1]{\emph{#1}}
\newcommand\ensurecomma{%
  \@ifnextchar,{}{\@latex@error{Don’t forget the comma!}{}}}
\newcommand\eg{\foreign{e.g.}\ensurecomma}
\newcommand\ie{\foreign{i.e.}\ensurecomma}
\newcommand\cf{\foreign{cf.\@}}

\newcommand\ensuresingleperiod{\@ifnextchar.{}{.\@}}
\newcommand\etc{\foreign{etc}\ensuresingleperiod}
\newcommand\etal{\foreign{et al}\ensuresingleperiod}

\DeclareGraphicsExtensions{.eps,.pdf,.png,.gif,.jpg}

\newcounter{LabPhase}
\setcounter{LabPhase}{0}%

\newenvironment{LabExercises}{%
\renewcommand{\ExerciseListName}{Question}%
\renewcommand{\ExerciseListHeader}{\textbf{%
   Lab\ExerciseHeaderNB. }}
\begin{ExerciseList}}%
{\end{ExerciseList}}
\newcommand{\Lab}{\Exercise[name={Lab Phase\ExerciseHeaderNB},counter={LabPhase}]}
\newcounter{CheckPoint}
\setcounter{CheckPoint}{1}%

\newcommand{\Checkpoint}{\textbf{Checkpoint \theCheckPoint }:: \addtocounter{CheckPoint}{1}}

\newcounter{myQuestion}
\setcounter{myQuestion}{0}

\newcommand{\myQuestion}{\addtocounter{myQuestion}{1}\emph{Question \themyQuestion}-- }


\lstset{language=java,
  basicstyle=\small\ttfamily,
  numbers=none, 
  numberstyle=\tiny, 
  stepnumber=1, 
  numbersep=5pt,
  frame=single,
  captionpos=b,
  rulecolor=\color{nicered}
}

\lstdefinelanguage{cline}
{
  morecomment=[s][\color{commandpromptforeground}]{ }{ },
} 

\lstnewenvironment{commandline}[1][]
  {\lstset{language=cline,numbers=none,frame=none,backgroundcolor=\color{commandlinebackground},basicstyle=\color{commandlineforeground},nolol,#1}}
  {}

\setlength{\hoffset}{0in}
\setlength{\voffset}{0in}
\settypeblocksize{9.5in}{7.5in}{*}
\setlrmargins{0.5in}{*}{1}
\setulmargins{0.75in}{*}{*}
\setheadfoot{\onelineskip}{2\onelineskip}
\checkandfixthelayout

\begin{document}

\begin{center}
\Large{CIS 201 Fall 2008\\
Lab Postmortem By Lab}
\end{center}


\begin{LabExercises}
\Lab Logging in and typing in a FANG Program.

Overall successful. Need to make pair programming more explicit. Also
need to make sure both students include their names in the program
file and both end up with a copy; may want to modify Moodle to assign
pairs weekly.

\Lab Catnip - Animation and moving with keyboard.

Generally successful.

\Lab Orbits - Expressions and units.

Generally successful. Not sure students understand the mathematics.

\Lab Racing - Conditionals in moving.

Not successful. (1) Line/object intersection is broken in FANG (on
list to fix). (2) Car would be better implemented as a narrow triangle
so a front/rear are apparent from the appearance. The forward()
command makes sense. 

\Lab RakingLeaves - An \code{ArrayList} of leaves.

Generally successful if a bit complex. The program has too many
parts. It would make sense to develop in class a \code{randomColor()}
method and then use that method in the game (interconnects the two
parts of the class). Either provide the code for making the pile or
the code for creating the leaves in the playing field; that would
remove some complexity and provide model method comments. 

\Lab StrSpr - Make your own \code{StringSprite}

\Lab EasyDice - Provide a class for a program.

  Too long. No questions for the students to answer (probably good
  given the length but the point is that no questions means no
  reflection so the learning, if any, doesn't stick very well). 

  \begin{enumerate}
  \item Remove the changing of the size/color of the program in
    the provided \fname{EasyDice.java} file. Students should not change
    \code{EasyDice} at all.
  \item Provide some examples of the stages of development. Perhaps a
    stubbed version of the \code{OneDie} class. This is a bit much but
    they have no experience with this yet.
  \item Alternatively, provide complete header comments with notes on
    \emph{how} the code should work as well as the contract. That
    would guide them in setting up fields and methods.
  \end{enumerate}

\Lab Tic Tac Toe - Game components with state. 

  Too long but generally a good assignment. Students figure out how to
  map one-dimensional array onto a two-dimensional layout on the
  screen. Design questions are good; need to make it more clear that
  they must turn them in.

\Lab Bouncing Bumpers - Reading a structured text file.

  Too long. Good idea but needs to provide more structure for the
  students. Having them write their own data file is a good idea.

\end{LabExercises}

\large{Other notes}

Need to work on types much more. 

Choral response for ``to declare a variable to hold an arbitrary
number of items of type X'' you should be thinking ``ArrayList''.

\end{document}

%%% Local Variables: 
%%% mode: latex
%%% End: 

% LocalWords:  Moodle Ladd's login emacs
