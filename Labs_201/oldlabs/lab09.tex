% book.tex -- Copyright (c) 2004, Brian C. Ladd
%
%  Author : Brian C. Ladd 
%  Created: Wed Jul 07 12:03:49 2004 by blad
%  Revised: Fri Dec 28 15:27:43 2007 by blad
%  
\def\FileCreated{Wed Jul 07 12:03:49 2004}
\def\FileRevised{Fri Dec 28 15:27:43 2007}
\documentclass[12pt,oneside]{memoir}
\usepackage{epsfig}
\usepackage{graphicx}
\usepackage[usenames]{color}
\usepackage{xltxtra}
\usepackage{fontspec}
\usepackage{textcomp}
\usepackage{pstricks}
\usepackage{listings}[2000/08/23] 

\usepackage{rotating}
\usepackage{exercise}
% ----- Fonts -----
\defaultfontfeatures{Scale=MatchLowercase,Mapping=tex-text}
\setmainfont{Gentium Book Basic}
\setmonofont{Bitstream Vera Sans Mono}
\setsansfont{Arial}
% ----- Fonts -----

\definecolor{nicered}{rgb}{.647,.129,.149}
\definecolor{listinggray}{gray}{0.1}
\definecolor{templategrey}{gray}{0.85}
\definecolor{NewtonsApple}{gray}{.95}
\definecolor{commandlinebackground}{gray}{.90}
\definecolor{commandlineforeground}{gray}{0.20,}
\definecolor{commandpromptforeground}{gray}{0.55}

\newcommand\code[1]{\lstinline^#1^}
\newcommand\fname[1]{\texttt{#1}}
\newcommand\note[1]{\unskip\footnote{#1}}
\newcommand\foreign[1]{\emph{#1}}
\newcommand\ensurecomma{%
  \@ifnextchar,{}{\@latex@error{Don’t forget the comma!}{}}}
\newcommand\eg{\foreign{e.g.}\ensurecomma}
\newcommand\ie{\foreign{i.e.}\ensurecomma}
\newcommand\cf{\foreign{cf.\@}}

\newcommand\ensuresingleperiod{\@ifnextchar.{}{.\@}}
\newcommand\etc{\foreign{etc}\ensuresingleperiod}
\newcommand\etal{\foreign{et al}\ensuresingleperiod}

\DeclareGraphicsExtensions{.eps,.pdf,.png,.gif,.jpg}

\newcounter{LabPhase}
\setcounter{LabPhase}{0}%

\newenvironment{LabExercises}{%
\renewcommand{\ExerciseListName}{Question}%
\renewcommand{\ExerciseListHeader}{\textbf{%
   Phase\ExerciseHeaderNB. }}
\begin{ExerciseList}}%
{\end{ExerciseList}}
\newcommand{\LabExercise}{\Exercise[name={Lab Phase\ExerciseHeaderNB},counter={LabPhase}]}
\newcounter{CheckPoint}
\setcounter{CheckPoint}{1}%

\newcommand{\Checkpoint}{\textbf{Checkpoint \theCheckPoint }:: \addtocounter{CheckPoint}{1}}


\lstset{language=java,
  basicstyle=\small\ttfamily,
  numbers=none, 
  numberstyle=\tiny, 
  stepnumber=1, 
  numbersep=5pt,
  frame=single,
  captionpos=b,
  rulecolor=\color{nicered}
}

\lstdefinelanguage{cline}
{
  morecomment=[s][\color{commandpromptforeground}]{ }{ },
} 

\lstnewenvironment{commandline}[1][]
  {\lstset{language=cline,numbers=none,frame=none,backgroundcolor=\color{commandlinebackground},basicstyle=\color{commandlineforeground},nolol,#1}}
  {}

\setlength{\hoffset}{0in}
\setlength{\voffset}{0in}
\settypeblocksize{9.5in}{7.5in}{*}
\setlrmargins{0.5in}{*}{1}
\setulmargins{0.75in}{*}{*}
\setheadfoot{\onelineskip}{2\onelineskip}
\checkandfixthelayout

\begin{document}

\begin{center}
\Large{CIS 201 Fall 2008: Lab 9\\Bouncing Bumpers}
\end{center}

\begin{itemize}
\item Reading a file
\end{itemize}

Consider the process of writing a pinball game. Given a screen and a
ball that bounces, how do different levels or pinball games differ?
They differ in how their bumpers (and other things) are laid out. 

What we really want, then, is to build a pinball \emph{engine}, a
program that can simulate any pinball machine described in an
appropriate way. You will begin with a simulator that has ball physics
but no bumpers. Your job is to extend the program so that it prompts
the user for the name of a bumper file (format described below) and
then creates the appropriate sprites in the machine before setting the
ball(s) loose.

\begin{LabExercises}
\LabExercise Download the files \fname{Pinball.java} and
\code{PinballGame.java} into a directory for \fname{Lab9}. The two
files are linked through Moodle (in the description for the Lab where
you will turn it in) and are located at
\fname{http://cs.potsdam.edu/faculty/laddbc/Teaching/CS1/Week11/Pinball.java}
and
\fname{http://cs.potsdam.edu/faculty/laddbc/Teaching/CS1/Week11/PinballGame.java}
respectively.

Also in that directory are several \fname{.pnb} files; \fname{.pnb} is
an extension I coined to hold pinball description files. It is a text
format described further down.

Download the Java and pinball files into your lab directory. Compile
and run the pinball game. It should compile but it should not do
anything because there is no description of a pinball game. 

\LabExercise Look at the \code{PinballGame} class. Find all of the
stub functions (they are the ones with no code in them). You are going
to need to replace the stubs with working code.

\LabExercise Modify \code{getFileName} so that it displays the prompt
that is passed into it followed by a single space and then reads a
string from the user, returning that value. The returned value should
be permitted to include spaces. Modify the \code{connectScannerToFile}
method so it prints out the name of the file it was passed (this is
still stub code).

\Checkpoint Compile and run your program and show one of the
instructors your program running (prompt, read, echo name of file).

\LabExercise Now you will need to replace the method
\code{ensureExtension}.

\code{ensureExtension} is passed a file name and returns that file
name with a given extension. That is, if it were passed
\code{something} and \code{txt}, it would return the string
\code{something.txt}. The method is called \emph{ensure}Extension
because it should only add the extension if the file name does not
already end with it. Thus calling the method with \code{something.txt}
and \code{txt} should also return \code{something.txt}.

Your method will take two \code{String} parameters. You will want to
look at the documentation for the \code{String} class (the
\code{substring}, \code{equals}, \code{endsWith} methods in
particular). The first parameter is the file name, the second the
extension. If the file name does not end with a dot and the given
extension, it needs fixing; otherwise return it unchanged. To fix a
file name, if it ends with a dot, just return file name plus
extension; otherwise return file name plus dot plus extension.

After you get the file name (from \code{getFileName} above) but before
passing it to \code{connectScannerToFile}, you should ensure that it
has the extension for a pinball description file; something like this:

\begin{lstlisting}
fname = ensureExtension(fname, "pnb");
\end{lstlisting}


\LabExercise Now, read the whole file. You will write a loop that
reads the next token from the file (a token is, in this case, a
word). Then, if the word is ``Ball'', ``Oval'', or ``Rectangle'' you
will call a routine that reads the rest of the information for that
type of sprite and adds one to the scene.

That is, in \code{readBufferFile} you will use the provided code to
prompt the user and get a \code{Scanner} hooked up to a file. Then,
while there is another word to read, read it. If the word is ``Ball'',
call the \code{handlePinball} method (and so on for the other two
types). Lines are of the form:

\begin{lstlisting}
Ball <color> <x> <y> <w> <h> <dX> <dY>
Oval <color> <x> <y> <w> <h>
Rectangle <color> <x> <y> <w> <h>
\end{lstlisting}

Where \code{<color>} is a word that can be passed to \code{getColor}
to return a color and all the rest are double values. x and y are the
location of the bumper (or ball); w and h are width and height. The dX
and dY are the velocity of the ball. You should use the \code{Scanner}
to do most of the work of reading all the fields of a given bumper.

\Checkpoint Show your code reading files to one of the instructors.

\LabExercise Write a pinball file that describes a pinball level with
two balls (starting near the left and right edges of the screen
heading toward the center of the screen) and two oval sprite eyes and
one rectangle sprite ``mouth'' to make a ``smiley face''.

\LabExercise Upload your Java and text files.

Make sure both authors' names are in all of the header comments!

Go to Moodle; you will find an assignment in week 11 labeled
\textbf{Lab 9}. Go there and upload all of the Java files you
created/modified in lab today. Note that there will be other files in
your \fname{Lab8} directory. You want to make sure you upload just the
\fname{.java} file.

Also make sure that both partners have copies of the program. This
program has some bearing on the current assignment so you probably
want to have this code around. Comments will also be very helpful in
this program for exactly that reason.

\end{LabExercises}


\end{document}

%%% Local Variables: 
%%% mode: latex
%%% End: 

% LocalWords:  Moodle Ladd's login emacs
