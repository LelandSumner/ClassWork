\def\FileCreated{Wed Jul 07 12:03:49 2004}
\def\FileRevised{Fri Dec 28 15:27:43 2007}
\documentclass[12pt,oneside]{memoir}
\usepackage{epsfig}
\usepackage{graphicx}
\usepackage[usenames]{color}
\usepackage{xltxtra}
\usepackage{fontspec}
\usepackage{textcomp}
\usepackage{pstricks}
\usepackage{listings}[2000/08/23] 

\usepackage{rotating}
\usepackage{exercise}
% ----- Fonts -----
\defaultfontfeatures{Scale=MatchLowercase,Mapping=tex-text}
\setmainfont{Gentium Book Basic}
\setmonofont{Bitstream Vera Sans Mono}
\setsansfont{Arial}
% ----- Fonts -----

\definecolor{nicered}{rgb}{.647,.129,.149}
\definecolor{listinggray}{gray}{0.1}
\definecolor{templategrey}{gray}{0.85}
\definecolor{NewtonsApple}{gray}{.95}
\definecolor{commandlinebackground}{gray}{.90}
\definecolor{commandlineforeground}{gray}{0.20,}
\definecolor{commandpromptforeground}{gray}{0.55}

\newcommand\code[1]{\lstinline^#1^}
\newcommand\fname[1]{\texttt{#1}}
\newcommand\note[1]{\unskip\footnote{#1}}
\newcommand\foreign[1]{\emph{#1}}
\newcommand\ensurecomma{%
  \@ifnextchar,{}{\@latex@error{Don’t forget the comma!}{}}}
\newcommand\eg{\foreign{e.g.}\ensurecomma}
\newcommand\ie{\foreign{i.e.}\ensurecomma}
\newcommand\cf{\foreign{cf.\@}}

\newcommand\ensuresingleperiod{\@ifnextchar.{}{.\@}}
\newcommand\etc{\foreign{etc}\ensuresingleperiod}
\newcommand\etal{\foreign{et al}\ensuresingleperiod}

\DeclareGraphicsExtensions{.eps,.pdf,.png,.gif,.jpg}

\newcounter{LabPhase}
\setcounter{LabPhase}{0}%

\newenvironment{LabExercises}{%
\renewcommand{\ExerciseListName}{Question}%
\renewcommand{\ExerciseListHeader}{\textbf{%
   Phase\ExerciseHeaderNB. }}
\begin{ExerciseList}}%
{\end{ExerciseList}}
\newcommand{\LabExercise}{\Exercise[name={Lab Phase\ExerciseHeaderNB},counter={LabPhase}]}
\newcounter{CheckPoint}
\setcounter{CheckPoint}{1}%

\newcommand{\Checkpoint}{\textbf{Checkpoint \theCheckPoint }:: \addtocounter{CheckPoint}{1}}

\newcounter{myQuestion}
\setcounter{myQuestion}{0}

\newcommand{\myQuestion}{\addtocounter{myQuestion}{1}\emph{Question \themyQuestion}-- }


\lstset{language=java,
  basicstyle=\small\ttfamily,
  numbers=none, 
  numberstyle=\tiny, 
  stepnumber=1, 
  numbersep=5pt,
  frame=single,
  captionpos=b,
  rulecolor=\color{nicered}
}

\lstdefinelanguage{cline}
{
  morecomment=[s][\color{commandpromptforeground}]{ }{ },
} 

\lstnewenvironment{commandline}[1][]
  {\lstset{language=cline,numbers=none,frame=none,backgroundcolor=\color{commandlinebackground},basicstyle=\color{commandlineforeground},nolol,#1}}
  {}

\setlength{\hoffset}{0in}
\setlength{\voffset}{0in}
\settypeblocksize{9.5in}{7.5in}{*}
\setlrmargins{0.5in}{*}{1}
\setulmargins{0.75in}{*}{*}
\setheadfoot{\onelineskip}{2\onelineskip}
\checkandfixthelayout

\begin{document}

\begin{center}
\Large{CIS 201 Fall 2008: Lab 11\\Caesar Cipher}
\end{center}

\begin{itemize}
\item Non-FANG Java
\item Reading a file
\item Writing a file
\end{itemize}

This lab will take the file copying program from the last lab and
extend it so that instead of just copying text from one file to
another, it will modify each character and then print out the modified
version of the text. That is, after reading a line, the line will be
modified, character by character, into a new line. The new line will
then be printed in place of the original line.

Why would you want to do something like this? Consider a text file
containing something like the text for this assignment. Your program
from last week could copy it but what if we wanted the copy to add
terms to a search database (think \textbf{Google}). We might want to
put all terms into the database in a known case (so that ``mark'',
``Mark'', and ``MARK'' are all stored as the same word). Converting
all letters in the program to capital letters would be a good start. 

Now, if you have been reading Java documentation, you might remember
that there is a \code{String} method to convert case on a whole string
at a time; you will \emph{not} be using that method. Instead, you will
build that method. Then, with just a little modification, you will
convert your capitalizing program into a simple encryption/decryption
tool.

At the heart of both programs is a nested loop that looks something
like this (this is \textbf{pseudocode}; it will not compile but you
should be able to see what is happening):

\begin{lstlisting}
while (there are more lines) {
  String original = nextLine;
  String converted = "";
  while (there are characters  in original) {
    char nextCh = nextChar;
    add nextCh to end of converted;
  }
  print out converted;
}
\end{lstlisting}

\begin{LabExercises}
  \LabExercise Create a new folder and copy over one partner's copy of
  \fname{CopyTextFile.java} (from the previous lab). 

  Change the name of the copy to \fname{CapitalizeCopy.java}. Modify
  the program and get it to compile.

  \myQuestion What is an \emph{end-of-file-controlled loop}? Where
  would you find one in \fname{CapitalizeCopy.java}?
  
  \LabExercise Modify the EOF-controlled loop in the program to create
  a character-by-character copy of the line read.

  Look at the pseudocode above. There needs to be a loop inside of the
  EOF-controlled loop that converts each and every character in the
  \code{String}. If we have to do something to every character in a
  \code{String}, what should you be thinking?

  You have no idea how to answer that, do you? That is, if it were an
  array or even an \code{ArrayList} you would know to think, ``Use a
  \code{for} loop.'' 

  Since we know, before the loop starts, exactly how many times we
  need to do it (\code{String.length()} is a method that gives
  us...yep, the \emph{length} of the string), you should think of
  using a count-controlled \code{for} loop. 

  Right after declaring \code{line} inside of the EOF-controlled loop,
  declare a new \code{String} (initialized to the empty string). Add
  after that a loop to extract each character from \code{line} using
  an index, \code{i}. 

  Oops, another case where you don't know \emph{how} to do that. How
  to extract an element from a \code{String} given an index, \code{i}.

  \myQuestion How would you extract an element from an
  \code{ArrayList}, \code{theArrayList}, given an index \code{i}?

  \myQuestion How would you extract an element from an \emph{array},
  \code{theArray}, given an index \code{i}?

  \code{String} is an object type (how do I know that just by reading
  the name? Not a question for the lab but a sure-fire question on a
  final exam). That means I can use the dot notation to call a method
  on it. So it is like the \code{ArrayList} answer you gave above. To
  print out each character in \code{line}, one per line, on standard
  output I could:

  \begin{lstlisting}
for (int characterNdx = 0; characterNdx < line.length(); ++characterNdx) {    
  char nextCh = line.charAt(characterNdx);
  System.out.println(nextCh);
}
  \end{lstlisting}

  Instead of printing each character to standard output, your loop
  should append each character to the end of the converted string.

  Finally, you should print out to the file the converted string
  
  Compile and text your program; it should just copy files like
  before. The next phase will convert the case of characters.

  \Checkpoint Show us your code.

  \LabExercise Write five methods to handle characters.

  You will need to write five conversion methods. Each takes a single
  character as a parameter and returns a single character as its
  results. The reason there are five of them are one for each
  ``class'' of characters (lowercase letter, uppercase letter, digit,
  other) and one ``dispatch'' method which figures out the class of
  the character and calls the right one of the other four. So, let's
  start by writing five stub methods which just return their
  parameter:

  \begin{lstlisting}
// the dispatch method
public char convertCh(char ch) {
  return ch;
}

public char convertLC(char lower) {
  return lower;
}

public char convertUC(char upper) {
  return upper;
}

public char convertDigit(char digit) {
  return digit;
}

public char convertOther(char other) {
  return other;
}
  \end{lstlisting}

  Next, modify the character processing loop so that instead of just
  appending \code{nextCh}, you append the result of \code{convertCh}
  called with \code{nextCh}.

  Fill in the dispatch method according to the following pseudocode:

  \begin{lstlisting}
if (ch is lowercase letter) {
  return convertLC(ch);
} else if (ch is uppercase letter) {
  return convertUC(ch);    
} else if (ch is digit) {
  return convertDigit(ch);    
} else {
  return convertOther(ch);    
}
  \end{lstlisting}

  Remember: you need to know only three things about the ASCII
  character encoding: lowercase letters are contiguous; uppercase
  letters are contiguous; digits are contiguous.

  After filling in the dispatch routine, compile and test your
  program. It should still make unchanged copies. So, what is the last
  step for converting lowercase letters to uppercase letters? We must
  get the code for the character, add to it the difference between the
  uppercase and lowercase alphabets, and interpret the resulting
  number as a char.

  Huh?

  Characters are stored as numbers. We want that number for our
  lowercase letter, say 'a'. We need to convert 'a' to 'A' so we need
  to get the code for 'A'. To get there we just add the difference
  between the codes for 'A' and 'a'. So, if the code for 'a' were 100 and
  the code for 'A' were 150, we would have the following equation:

  \begin{lstlisting}
int lowercaseCode = 'a'; lowercaseCode = 100;
int littleaCode = 'a'; // littleaCode = 100;
// convert ASCII code to a number from 0-25; 0 for a, 1 for b, 2 for c
int alphabetIndex = lowercaseCode - littleaCode; // 100 - 100 = 0

int bigACode = 'A'; // bigACode = 150;
int resultCode = alphabetIndex + bigACode; // now code for capital letter
char result = (char)resultCode; // 150, treated as a char, or 'A'    
  \end{lstlisting}

  Read that again. Make sure you understand where everything came
  from. Now, that works for 'a'. What about 'b'? Lets look at it again
  with a different char:

  \begin{lstlisting}
int lowercaseCode = 'b'; lowercaseCode = 101;
int littleaCode = 'a'; // littleaCode = 100;
// convert ASCII code to a number from 0-25; 0 for a, 1 for b, 2 for c
int alphabetIndex = lowercaseCode - littleaCode; // 101 - 100 = 1

int bigACode = 'A'; // bigACode = 150;
int resultCode = alphabetIndex + bigACode; // now code for capital letter
char result = (char)resultCode; // 151, treated as a char, or 'B'    
  \end{lstlisting}

  \myQuestion How did I know the character code for 'b' and 'B'?

  \myQuestion How do I know that this works for any lowercase letter?

  So, you just need to change the body of \code{convertLC} to do the
  work above (without the comments because all of the numbers in them
  are fictional). 

  Compile, run, and test your program.

  \Checkpoint Show us your working program.

  \LabExercise Copy \code{CapitalizeCopy.java} to
  \code{EncipherCopy.java}. Modify it so it compiles and runs.

  \LabExercise Modify the code so that every letter is enciphered
  using a modified Caesar cipher.

  What is the Caesar cipher? It is where 'A' is converted to 'C', 'B'
  to 'D', and so on. It is a rotation of the alphabet. You will be
  writing code for a modified Caesar cipher: your code will move 'A'
  to the third letter in your last name (pick one author; if the
  letter is 'A' or your last name is too short, use 'F').

  Think for a moment: What methods must you change to convert all of
  the letters according to your Caesar cipher?

  To convert one letter: convert to an alphabet index (as in the above
  code), add the offset to your new first letter, and convert back to
  a letter. Does this make sense?

  \begin{lstlisting}
'd'-cipher: good dog -> jrrg grj
Plaintext:       g    o    o    d         d    o   g
Alphabet Index:  6   14   14    3    X    3   14   6
Shifted Index:   9   17   17    6         6   17   9
Ciphertext:      j    r    r    g         g    r   j
  \end{lstlisting}

  \myQuestion Encipher ``Happy Holidays'' using the 'd'-cipher.

  Okay, what happened when you enciphered 'y'? 'y' -> 24, add three
  and you get 27. This is a problem. What to do? Wait, 'y',
  \emph{rotated} forward 3 spots is 'b', right? And 'b' has an
  alphabet index of 1. And 27 \emph{modulo} 26 is...what is it?

  \code{27 \% 26 = ?} What does it equal? Modulus gives the remainder
  so it would be 1. Which is the alphabet index of 'b'. So, do the
  arithmetic \emph{modulo} 26.

  So, you figure out the alphabet index of the letter (as in the above
  code), add the appropriate offset, modulo 26, and then add back the
  code for the appropriate letter 'a'.

  \Checkpoint Show us your code working at \emph{enciphering} text
  files according to your modified Caesar cipher. 

  \LabExercise Copy \fname{EncipherCopy.java} to
  \fname{DecipherCopy.java}. Modify it so it compiles and runs.

  \LabExercise Modify \code{DecipherCopy} so that instead of
  \emph{enciphering} the file it \emph{deciphers} the file. How
  would you undo the ciphering operation?

  If you said ``subtraction'' and thought of the convert methods,
  you're right. The only problem is that we need non-negative numbers
  and the \code{\%} operator gives both negative and positive
  numbers. (Try printing out the value of (-1 \% 26). We would like the
  value to be 25 (one before 0) but it comes out to -1.) That should
  not be a problem because we can always add any multiple of 26 to our
  results (before the modulo operation) without changing the answer
  (after the modulo operation). So, subtract your shift value but add
  26 to the results.

  \LabExercise Upload your Java and text files.

  Make sure both authors' names are in all of the header comments!
  Make sure you answered all \themyQuestion {} questions in the header
  comments of the Java files you're uploading.

  Go to Moodle; you will find an assignment in this week for the
  lab. Go there and upload all of the \fname{.java} files you created
  today.

  Also make sure that both partners have copies of the program. This
  program has some bearing on the current assignment so you probably
  want to have this code around. Comments will also be very helpful in
  this program for exactly that reason.
\end{LabExercises}

\end{document}

%%% Local Variables: 
%%% mode: latex
%%% End: 

% LocalWords:  Moodle Ladd's login emacs
