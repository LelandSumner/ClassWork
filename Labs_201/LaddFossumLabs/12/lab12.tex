% book.tex -- Copyright (c) 2004, Brian C. Ladd
% 
% Author : Brian C. Ladd 
% Created: Wed Jul 07 12:03:49 2004 by blad
% Revised: Fri Dec 28 15:27:43 2007 by blad
% 
\def\FileCreated{Wed Jul 07 12:03:49 2004}
\def\FileRevised{Fri Dec 28 15:27:43 2007}
\documentclass[12pt,oneside]{memoir}
\usepackage{epsfig}
\usepackage{graphicx}
\usepackage[usenames]{color}
\usepackage{xltxtra}
\usepackage{fontspec}
\usepackage{textcomp}
\usepackage{pstricks}
\usepackage{listings}[2000/08/23] 

\usepackage{rotating}
\usepackage{exercise}
\definecolor{nicered}{rgb}{.647,.129,.149}
\definecolor{listinggray}{gray}{0.9}
\definecolor{templategrey}{gray}{0.30}
\definecolor{commandlinebackground}{gray}{0.25}
\definecolor{commandlineforeground}{gray}{0.85}
\definecolor{commandpromptforeground}{gray}{0.55}
% ----- Fonts -----
\defaultfontfeatures{Scale=MatchLowercase,Mapping=tex-text}
\setmainfont{Gentium Book Basic}
\setmonofont{Inconsolata}
\setsansfont{Arial}
% ----- Fonts -----
\newcommand\code[1]{\lstinline^#1^}
\newcommand\fname[1]{\texttt{#1}}

\lstset{language=java,
  basicstyle=\small\ttfamily,
  numbers=none, 
  numberstyle=\tiny, 
  stepnumber=1, 
  numbersep=5pt,
  frame=single,
  captionpos=b,
  rulecolor=\color{nicered}
}

\lstdefinelanguage{cline}
{
  morecomment=[s][\color{commandpromptforeground}]{ }{ },
} 

\lstnewenvironment{commandline}[1][]
{\lstset{language=cline,numbers=none,frame=none,backgroundcolor=\color{commandlinebackground},basicstyle=\color{commandlineforeground},nolol,#1}}
{}

\newenvironment{Checkpoint}[1]{%
  \begin{Exercise}[name={Checkpoint},title={#1}]}{%
  \end{Exercise}%
  \textbf{Show your work on Checkpoint~\theExercise{} to the lab monitor, %
    answering any necessary questions for them.  Have them sign before continuing.}}

\newcommand{\artDir}{../art}
\newcommand{\lab}[2]{%
  \title{CIS 201 Computer Science I\\Fall 2009 Lab #1\\#2}%
  \maketitle%
}

\setlength{\hoffset}{0in}
\setlength{\voffset}{0in}
\settypeblocksize{9.5in}{7.5in}{*}
\setlrmargins{0.5in}{*}{1}
\setulmargins{0.75in}{*}{*}
\setheadfoot{\onelineskip}{2\onelineskip}
\checkandfixthelayout

\begin{document}

\lab{12}{Reading Files}

\begin{itemize}
\item Working with files (for input and output).
\item Working with \code{String}s.
\item Different kinds of loops.
\end{itemize}

This lab will take the file copying program from the last lab and
extend it so that instead of just copying text from one file to
another, it will modify each character and then print out the modified
version of the text. That is, after reading a line, the line will be
modified, character by character, into a new line. The new line will
then be printed in place of the original line.

Why would you want to do something like this? Consider a text file
containing something like the text for this assignment. Your program
from last week could copy it but what if we wanted the copy to add
terms to a search database (think \textbf{Google}). We might want to
put all terms into the database in a known case (so that ``mark'',
``Mark'', and ``MARK'' are all stored as the same word). Converting
all letters in the program to capital letters would be a good start. 

Now, if you have been reading Java documentation, you might remember
that there is a \code{String} method to convert case on a whole string
at a time; you will \emph{not} be using that method. Instead, you will
build that method. Then, with just a little modification, you will
convert your capitalizing program into a simple encryption/decryption
tool.

\begin{Checkpoint}{ViewFileII.java}
Write a program that works just like \fname{ViewFile.java} in the last
lab \emph{except} that it copies each input line, character by
character, into another string and prints the \textbf{copy} to the
screen. You should start with \fname{EncipherFile.java} from last lab
and remove all character changes.

The new program should
\begin{itemize}
\item Prompt the user for an input file name and read in the file name
  from standard input. (Consider that you must prompt the user for
  multiple file names soon; how can you keep from repeating yourself?)
\item Open the file for input.
\item If there was a problem opening the file for input, provide a
  useful error message to the user. Be sure to include a description
  of the problem and the name of the offending file. Terminate the
  program after printing the error message.
\item If the file opened properly, copy the lines of the file to
  standard output; remember that \code{System.out} is a
  \code{PrintStream} and use a variable for output.
\item You can test your program by viewing the \fname{.java} file of
  the program itself.
\end{itemize}

Copying the contents of one \code{String} into another should not be
too difficult. The following code belongs inside the loop that reads
every line from the named input file. The lines (unmodified) are read
into \code{line} and the code copies them to \code{modifiedLine}. Your
code should then print the modified line out to the
\code{PrintStream} (that is hooked to the screen this time).

  \begin{lstlisting}
String line;
// read the next line into line (inside the hasNextLine loop)...
String modifiedLine = "";
for (int characterNdx = 0; 
     characterNdx < orginal.length(); 
     ++characterNdx) {    
  char nextCh = orginal.charAt(characterNdx);
  // this is where you convert nextCh...
  modifiedLine = modifiedLine + Character.toString(nextCh);
}
// now print out modifiedLine
  \end{lstlisting}
\end{Checkpoint}

\begin{Checkpoint}{SwitchCase.java}
Start with a copy of \fname{ViewFileII.java}. Change the name to
\code{SwitchCase} and make it compile.

Modify the program to copy one file to another (ask for two file
names, read input line by line, and copy it character by character to
a second string before writing in the output file. As each character
is transferred from one string to the other, you are to change the
case of the letter. All uppercase letters should be lowercase and all
lowercase letters should be uppercase; everything else passes through
unchanged.

You may \emph{not} use any built-in Java \code{toUpper} or
\code{toUpperCase} method (or \code{toLower*}). 

Remember: you need to know only three things about the ASCII character
encoding: lowercase letters are contiguous; uppercase letters are
contiguous; digits are contiguous. How does that help you with this
assignment? 

You can determine whether the character being copied is an uppercase
letter, a lowercase letter, or neither using routines in the
\code{Character} class built into Java. Given that you have a
lowercase letter, how do you convert it to uppercase?

Get the code for the letter (the ASCII code), add the difference
between the uppercase and lowercase range of letters, and interpret
the result as a character.

Alternatively, get the code for the letter, determine how far the
letter is from the beginning of the lowercase alphabet, find the code
for the letter the same distance from the beginning of the uppercase
alphabet, and interpret that code as a character.

Huh?

Characters are stored as numbers. We want that number for our
lowercase letter, say 'g'. We need to convert 'g' to 'G' so we need
to get the code for 'a' and subtract it from 'g', giving us 6 (make
sure you believe this is the right number). Then add 6 to the code for
'A' and interpret it as a char, or as 'G' (why \emph{must} it be
'G'?).

Assuming, for the sake of argument, that 'a' has code 100 and 'A' has
code 150 (neither number is correct), we have the following equations:
just add the difference
between the codes for 'A' and 'a'. So, if the code for 'a' were 100 and
the code for 'A' were 150, we would have the following equation:

\begin{lstlisting}
int little_a = 'a';           // little_a is 100
int big_A = 'A';              // big_A is 150

char ch = 'g';                // the character to translate
int chCode = ch;              // get the code
                              // 1111111
                              // 0000000
                              // 0123456
                              // abcdefg - g = 106

int chOffset = 
  chCode - little_a;          // 106 - 100 = 6

int ucCode = 
  big_A + chOffset;           // 150 + 6 = 156
                              // 1111111
                              // 5555555
                              // 0123456
                              // ABCDEFG - 156 = G
char uc = (char) ucCode;
\end{lstlisting}

Read the code again. Make sure you understand where everything came
from. Now, that works for 'g'. What about 'n'? Work out, by hand, the
value of \code{chCode}, \code{chOffset}, \code{ucCode} and \code{uc}
(and show them to the lab monitor).

What changes to go the other way, to go from an uppercase letter to a
lowercase letter? The uppercase letter's offset is found off 'A'
and the lowercase letter's offset is found off 'a'. That is the only
change. Write the code and trace (by hand) what happens when the
current character is 'K'. Show that to the lab monitor, too.

Finally, put the translation code in to translate each letter before
adding it to the translated string and writing it out to the copied
file.
\end{Checkpoint}

\begin{Checkpoint}{Caesar.java}
Starting with a copy of \fname{SwitchCase.java}, change the name to
\code{Caesar} and make it compile.

Before writing the second file, each \emph{letter} in the input file
is to be enciphered using a modified Caesar cipher. In the Caesar
cipher, each letter is rotated forward some number of characters (all
letters move the same amount). So, if we work with the C-cipher, 'A'
goes to 'C', 'B' to 'D', and so on until 'X' to 'Z', 'Y' to 'A', and
'Z' to 'B'. Every letter goes forward \emph{modulo 26}.

You will map A to the first letter of the last name of the partner who
logged on \emph{unless} that letter is 'A' or 'M'. If both partners
have an 'A' or 'M', use the first letter of a first name.

Your modified program should 
\begin{itemize}
\item Prompt the user for an input file name and read in the file name
  from standard input. (Consider that you must prompt the user for
  multiple file names soon; how can you keep from repeating yourself?)
\item Open the file for input.
\item If there was a problem opening the file for input, provide a
  useful error message to the user. Be sure to include a description
  of the problem and the name of the offending file. Terminate the
  program after printing the error message.
\item Prompt the user for an output file name and read the  file name
  from standard input.
\item Open the file for output.
\item If there was a problem, give a reasonable error message and
  terminate the program.
\item If both files opened properly, copy the lines of the input file to
  the output file, enciphering each letter using your given Caesar
  cipher. Uppercase letters should encipher to uppercase letters;
  lowercase letters should encipher to lowercase letters; and
  everything else should pass through unchanged.
\item You can test your program by enciphering the \fname{.java} file of
  the program itself and checking a couple of the words.
\end{itemize}

  What is the Caesar cipher? It is where 'A' is converted to 'C', 'B'
  to 'D', and so on. It is a rotation of the alphabet. You will be
  writing code for a modified Caesar cipher: your code will move 'A'
  to the third letter in your last name (pick one author; if the
  letter is 'A' or your last name is too short, use 'F').

  Think for a moment: What methods must you change to convert all of
  the letters according to your Caesar cipher?

  To convert one letter: convert to an alphabet index (as in the above
  code), add the offset to your new first letter, and convert back to
  a letter. Does this make sense?

  \begin{lstlisting}
'd'-cipher: good dog -> jrrg grj
Plaintext:       g    o    o    d         d    o   g
Alphabet Index:  6   14   14    3    X    3   14   6
Shifted Index:   9   17   17    6         6   17   9
Ciphertext:      j    r    r    g         g    r   j
  \end{lstlisting}

  Encipher ``Happy Holidays'' using the 'd'-cipher.

  Okay, what happened when you enciphered 'y'? 'y' -> 24, add three
  and you get 27. This is a problem. What to do? Wait, 'y',
  \emph{rotated} forward 3 spots is 'b', right? And 'b' has an
  alphabet index of 1. And 27 \emph{modulo} 26 is...what is it?

  \code{27 \% 26 = ?} What does it equal? Modulus gives the remainder
  so it would be 1. Which is the alphabet index of 'b'. So, do the
  arithmetic \emph{modulo} 26.

  So, you figure out the alphabet index of the letter (as in the above
  code), add the appropriate offset, modulo 26, and then add back the
  code for the appropriate letter 'a'.
\end{Checkpoint}

\begin{Checkpoint}{UnCaesar.java}
Starting with \fname{Caesar.java}, you should rename it to
\code{UnCaesar.java} and get it to compile.

How do you \emph{decipher} a Caesar ciphered file? You modify each
letter but how?

If you said ``subtraction'' then you're right. The only problem is
that we need non-negative numbers and the \code{\%} operator gives
both negative and positive numbers. (Try printing out the value of (-1
\% 26). We would like the value to be 25 (one before 0) but it comes
out to -1.) That should not be a problem because we can add any
multiple of 26 to our results (before the modulo operation) without
changing the answer (after the modulo operation). So, subtract your
shift value but add 26 to the results.

Make the program decipher according to your Caesar cipher.

\end{Checkpoint}

\Large{Log off of the lab computer you are using before leaving they
  lab. Anyone entering the lab has unlimited access to your files if
  you remain logged on. \textbf{DO NOT} turn off lab computers! They
  are a shared resource and there might be someone else logged in to
  ``your'' machine.}
\end{document}

%%% Local Variables: 
%%% mode: latex
%%% End: 

% LocalWords:  Moodle Ladd's login emacs
