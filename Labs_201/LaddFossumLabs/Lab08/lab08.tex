\documentclass[12pt]{article}

\pagestyle{plain}
\oddsidemargin 0in       %   Left margin on odd-numbered pages.
\evensidemargin 0in        %   Left margin on even-numbered pages.
\textwidth 6.5in
\textheight 9in
\topmargin 0in
\headheight 0in
\headsep 0in

\usepackage[T1]{fontenc}
\usepackage{pslatex}
%\renewcommand*{\familydefault}{\sfdefault}

\usepackage{verbatim}
\newenvironment{qv}%
  {\quote
   \verbatim}%
  {\endverbatim
   \endquote}

\newcommand{\exx}{\\[2ex]}
\newcommand{\code}{\texttt}
\newcommand{\fname}{\texttt}
\newcommand{\eg}{{\em e.g.}}

\parskip 2ex
\parindent 0em


\begin{document}
\sloppypar

\begin{center}
\Large\bf
CIS 201 -- Computer Science I\\
Laboratory Assignment 8\\
\end{center}

\section*{Introduction}

In this lab you will redeem yourself
by writing programs to implement and/or test your exams problems.

Create a directory, \fname{Lab08} in your \fname{CS1} directory.
You will be creating Java programs in this directory
named \verb'Prob1.java', \verb'Prob2.java', and so forth,
corresponding to the questions on Exam 2.

\section*{Prob1}

Create a program named \verb'Prob1.java' with the contents
of problem 1 in your exam.
Instead of the class name \verb'Temp', use the class name \verb'Prob1'.

Compile and run this program.
Compare the output with what you gave on your exam.

\subsection*{Checkpoint 1}
{\bf 
Explain the output.
}

\section*{Prob2}

Create a program named \verb'Prob2.java' with the contents
of problem 2 in your exam.
Instead of the class name \verb'Temp1', use the class name \verb'Prob2'.

Compile and run this program.
Compare the output with what you gave on your exam.

\subsection*{Checkpoint 2}
{\bf 
Explain the output.
}

\section*{Prob3}

Create a program named \verb'Prob3.java'
with class name \verb'Prob3'
and with the prototypes given in problem 3 in your exam.
Don't worry about copying the documentation.

Implement the \verb'swap', \verb'randomInt', and \verb'shuffle' methods,
using what you did for your 15-Puzzle assignment.
The \verb'randomInt' and \verb'swap' methods
should be {\em almost} identical to what you were given
in the 15-Puzzle assignment, except that your \verb'swap' method
will use the formal array parameter
instead of the ArrayList field variable.

Write a \verb'main' method that tests your \verb'shuffle' method.
This should look almost the same as the \verb'main' method
in your \verb'Shuffle.java' program for your last assignment.
Be sure that you do more than one test,
with different array sizes (including an array of size zero!).

\subsection*{Checkpoint 3}
{\bf
Explain what you have done.
}

\section*{Prob4}

Create a program named \verb'Prob4.java'
with class name \verb'Prob4'.
Implement the \verb'intArrayToArrayList' method
and write a \verb'main' method
that tests your method.
The test should verify that the sizes of the given array
and ArrayList are identical
and that the equality expression holds.

\subsection*{Checkpoint 4}
{\bf
Show us your work.
}

\section*{Prob5, Prob6, and Prob7}

As you have done for the previous problems,
create appropriate Java files containing the appropriate class names.
Implement the method described in the exam problem
and write a \verb'main' method that tests your implementation.

\subsection*{Checkpoints 5, 6, and 7}
{\bf
Show us your work.
}

\end{document}
