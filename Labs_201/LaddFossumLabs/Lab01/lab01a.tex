\documentclass[12pt]{article}

\pagestyle{plain}
\oddsidemargin 0in       %   Left margin on odd-numbered pages.
\evensidemargin 0in        %   Left margin on even-numbered pages.
\textwidth 6.5in
\textheight 9in
\topmargin 0in
\headheight 0in
\headsep 0in

\usepackage[T1]{fontenc}
\usepackage{pslatex}

\usepackage{epsfig}
\usepackage{graphicx}

\parskip 2ex
\parindent 0em

\begin{document}
\sloppypar

\begin{center}
\Large\bf
CIS 201 -- Computer Science I\\
Laboratory Assignment 1a\\
\end{center}

\section*{Introduction}
\subsection*{Log into the Linux machine}

When you sit down in front of one of the Linux machines in the lab
(Dunn 358; also known as the ``Linux lab'', the ``CS Lab'' or, for our
purposes, just the ``lab''), you may have to press a key or move the
mouse to ``wake up'' the machine\footnote{or, if the machine is completely
turned off, power up the machine.}. Once the machine is running, you
will be faced with a login box.

You should type in your \emph{campus computing account} username and
password (the password will appear as dots so you will know how many
characters you have typed but no one can read your password over your
shoulder). Press the \texttt{Enter} key or press the \emph{Login}
button with your mouse.

You will then be logged into the machine. What you see now is the
\emph{XWindows Desktop}. Much if it is familiar from any other
operating system's desktop: there are windows, icons, a menus, and a
mouse pointer. The menu in the upper
left corner contains applications that you can run including the
\verb\Applications->Accessories->Terminal\
and the \verb\Applications->Internet->Iceape\ Web browser.

\subsection*{Accessing Moodle}

Moodle is a Web-based classroom management system. What that means is
that slides, schedules, notes, resources, grades, and even labs for
this course (or any course, for that matter) can be stored on
Moodle. In this phase you will navigate on the Web to Dr. Ladd's
Moodle pages, then to this class's Moodle site, create an account, and
log into the class. Then you will retrieve and print the remainder of
this lab.

The Iceape Web browser is installed on the Linux machines. You can
find it in the \verb\Applications->Internet\ entry in the menu. Launch the Web
browser.

The Web address of Moodle is:
\texttt{http://db.cs.potsdam.edu/moodle/}. On that page you will find
entries for several courses. You want to find \emph{CIS~201} (it
should be the first \textbf{Available Course}). Follow the link to
\emph{CIS~201}.

The login page for Moodle will come up. Since you do not yet have an
account, you want to click on the \emph{New Account} form on the
right-hand side of the screen. You will be prompted for a username,
password, and some personal information. Use your \emph{campus
  computing account} username as your username. Also use your
\emph{campus e-mail address} as your e-mail address. The name and
location fields should be straight forward. 

After you finish the form, a registration e-mail will be sent to your
e-mail. Fetch it (using your Bearmail web login)
and follow the instructions there in to finish
creating your account.

\subsection*{Checkpoint 1}
{\bf Show us your registration email message
from moodle before you continue.}

\section*{Completing your Moodle account login}

Now, once your account is created you will be prompted for an
``\emph{Enrolment Key}''. This key keeps random people from
across the world from enrolling in and seeing the contents of our
course. The key for this course is \texttt{satire} (one word, no
spaces). You should only be prompted for the key once at the beginning
of the semester.

You have now gotten into Moodle. You will see the course home
page. The the center area of the page, labeled \textbf{Weekly
  Outline} is where the course schedule,  assignments, 
readings, and other resources are found.

If you go to the first week (18 January -- 24 January), you will see a
link labeled \textbf{Writeup for Lab1b}.
Download that link, print the document,
and continue your lab from there.

\subsection*{Checkpoint 2}
{\bf Show us your printed Lab 1b document.}

\end{document}
