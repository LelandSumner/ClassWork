% book.tex -- Copyright (c) 2004, Brian C. Ladd
% 
% Author : Brian C. Ladd 
% Created: Wed Jul 07 12:03:49 2004 by blad
% Revised: Fri Dec 28 15:27:43 2007 by blad
% 
\def\FileCreated{Wed Jul 07 12:03:49 2004}
\def\FileRevised{Fri Dec 28 15:27:43 2007}
\documentclass[12pt,oneside]{memoir}
\usepackage{epsfig}
\usepackage{graphicx}
\usepackage[usenames]{color}
\usepackage{xltxtra}
\usepackage{fontspec}
\usepackage{textcomp}
\usepackage{pstricks}
\usepackage{listings}[2000/08/23] 

\usepackage{rotating}
\usepackage{exercise}
\definecolor{nicered}{rgb}{.647,.129,.149}
\definecolor{listinggray}{gray}{0.9}
\definecolor{templategrey}{gray}{0.30}
\definecolor{commandlinebackground}{gray}{0.25}
\definecolor{commandlineforeground}{gray}{0.85}
\definecolor{commandpromptforeground}{gray}{0.55}
% ----- Fonts -----
\defaultfontfeatures{Scale=MatchLowercase,Mapping=tex-text}
\setmainfont{Gentium Book Basic}
\setmonofont{Inconsolata}
\setsansfont{Arial}
% ----- Fonts -----
\newcommand\code[1]{\lstinline^#1^}
\newcommand\fname[1]{\texttt{#1}}

\lstset{language=java,
  basicstyle=\small\ttfamily,
  numbers=none, 
  numberstyle=\tiny, 
  stepnumber=1, 
  numbersep=5pt,
  frame=single,
  captionpos=b,
  rulecolor=\color{nicered}
}

\lstdefinelanguage{cline}
{
  morecomment=[s][\color{commandpromptforeground}]{ }{ },
} 

\lstnewenvironment{commandline}[1][]
{\lstset{language=cline,numbers=none,frame=none,backgroundcolor=\color{commandlinebackground},basicstyle=\color{commandlineforeground},nolol,#1}}
{}

\newenvironment{Checkpoint}[1]{%
  \begin{Exercise}[name={Checkpoint},title={#1}]}{%
  \end{Exercise}%
  \textbf{Show your work on Checkpoint~\theExercise{} to the lab monitor, %
    answering any necessary questions for them.  Have them sign before continuing.}}

\newcommand{\artDir}{../art}
\newcommand{\lab}[2]{%
  \title{CIS 201 Computer Science I\\Fall 2009 Lab #1\\#2}%
  \maketitle%
}

\setlength{\hoffset}{0in}
\setlength{\voffset}{0in}
\settypeblocksize{9.5in}{7.5in}{*}
\setlrmargins{0.5in}{*}{1}
\setulmargins{0.75in}{*}{*}
\setheadfoot{\onelineskip}{2\onelineskip}
\checkandfixthelayout

\begin{document}

\lab{09}{Experiments with Code}

\begin{itemize}
\item Understanding (and testing) loops
\item Thinking about dimensions in output
\end{itemize}

\begin{Checkpoint}{Getting Started}
  Rather than showing each running section of a single program to the
  lab monitor, in this lab you will read code, answer questions about
  the code, and show your answers to the lab monitor. Then, as part of
  the \emph{following} checkpoint you will type in and execute the
  program so that you can check your own answers.

  \begin{lstlisting}
    System.out.println("*******");
    System.out.println(" ***** ");
    System.out.println("  ***  ");
    System.out.println("   *   ");     
  \end{lstlisting}

  \begin{enumerate}
  \item How many stars does the above code print?
  \item How easily could it be changed to print 10 lines in the given
    pattern?
  \item How easily could it be changed to print 100 lines in the given
    pattern?
  \end{enumerate}
\end{Checkpoint}

\begin{Checkpoint}{Some Loops}
  Go ahead and type in a non-FANG program and include the above
  code. Run the program and verify your count.
  \begin{lstlisting}
    for (int i = 0; i != 10; ++i) {
      System.out.print(" " + i);
    }
    System.out.println();
    for (int j = 10; j != 0; --j) {
      System.out.print(" " + j);
    }
    System.out.println();    
  \end{lstlisting}

  \begin{enumerate}
  \item What is the output of the above code?
  \item What (if anything) is different between the ranges of the two
    loops? Why are they (or are they not) different?
  \end{enumerate}
\end{Checkpoint}

\begin{Checkpoint}{Nested Loops I}
  Go ahead and type in a non-FANG program and include the above
  code. Run the program and verify your count.

  \begin{lstlisting}
    for (int r = 0; r != 4; ++r) {
      for (int c = r; c != 0; --c) {
        System.out.print(" ");
      }
      for (int c = 3 - r; c != 0; --c) {
        System.out.print("*");
      }
      System.out.print("*");
      for (int c = 3 - r; c != 0; --c) {
        System.out.print("*");
      }      
      System.out.println();
    }
  \end{lstlisting}

  \begin{enumerate}
  \item How many stars does the above code print?
  \item In what shape is the output? Is there a pattern to the
    different lines?
  \item How difficult would it be to modify the code to print 30 lines
    in the given pattern? How would you do it?
  \end{enumerate}
\end{Checkpoint}

\begin{Checkpoint}{Parameters}
  Go ahead and type in a non-FANG program and include the above
  code. Run the program and verify your count.

  \begin{lstlisting}    
  public void p(int x) {
    for (int i = 0; i != x; ++i) {
      if ((i % 3) == 0)  {
        for (int j = 0; j != x; ++j) {
          System.out.print("*");
        }
        System.out.println();
      }
    }
  }
  \end{lstlisting}
  \begin{enumerate}
  \item How many \emph{rows} are drawn by a call of \code{p(10)}?
  \item How many \emph{columns} in the first row for the above call?
  \item How many stars are drawn by a call of \code{p(10)}?
  \item How many stars are drawn by a call of \code{p(3)}? and \code{p(4)}?
  \end{enumerate}
\end{Checkpoint}

\begin{Checkpoint}{os}
  Go ahead and type in a non-FANG program and include the above
  code. Run the program and verify your count.

  \begin{lstlisting}
  public static final int CUTOFF = 10;
  public static final int FULLWIDTH = 30;
    
  public void os(boolean stars) {
    for (int c = 0; c != FULLWIDTH; ++c) {
      if (stars && (c < CUTOFF)) {
        System.out.print("*");
      } else {
        System.out.print("-");
      }
    }
    System.out.println();
  }
  \end{lstlisting}

  \begin{enumerate}
  \item What does \code{os} print when the parameter is \code{true}?
  \item What does \code{os} print when the parameter is \code{false}?
  \item How difficult would it be to change where the two symbols
    change over?
  \end{enumerate}
\end{Checkpoint}

\begin{Checkpoint}{ts}
  Go ahead and type in a non-FANG program and include the above
  code. Run the program and verify your conclusions.

  \begin{lstlisting}
  public void ts(boolean stars) {
    for (int c = 0; c != FULLWIDTH; ++c) {
      if (stars && (c < CUTOFF)) {
        System.out.print("*");
      } else {
        System.out.print("-");
      }
    }
    System.out.println();
    for (int c = 0; c != FULLWIDTH; ++c) {
      if (stars && (c < CUTOFF)) {
        System.out.print("*");
      } else {
        System.out.print(" ");
      }
    }
    System.out.println();
  }    
  \end{lstlisting}
  
  \begin{enumerate}
  \item Using the same constants given in the previous question, what
    does \code{ts} do with a \code{true} or a \code{false} constant.
  \item How many stars are drawn when \code{ts(true)} is executed?
  \end{enumerate}
\end{Checkpoint}

\begin{Checkpoint}{what}
  Go ahead and type in a non-FANG program and include the above
  code. Run the program and verify your conclusions.

  \begin{lstlisting}
  public void what() {
    for (int tsCount = 0; tsCount != 6; ++tsCount) {
      ts(tsCount < 3);
    }
    os(false);
  }
  \end{lstlisting}

  \begin{enumerate}
  \item How many rows does this code write on the screen?
  \item How many times is \code{ts} called with \code{true}?
  \item How many times is \code{ts} called with \code{false}?
  \item How many stars are drawn?
  \item What does the output look like?
  \end{enumerate}
\end{Checkpoint}

\begin{Checkpoint}{}
  Go ahead and type in a non-FANG program and include the above
  code. Run the program and verify your conclusions.
\end{Checkpoint}


\Large{Log off of the lab computer you are using before leaving they
  lab. Anyone entering the lab has unlimited access to your files if
  you remain logged on. \textbf{DO NOT} turn off lab computers! They
  are a shared resource and there might be someone else logged in to
  ``your'' machine.}
\end{document}

%%% Local Variables: 
%%% mode: latex
%%% End: 

% LocalWords:  Moodle Ladd's login emacs
