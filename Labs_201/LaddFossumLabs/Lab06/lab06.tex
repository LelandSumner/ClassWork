\documentclass[12pt]{article}

\pagestyle{plain}
\oddsidemargin 0in       %   Left margin on odd-numbered pages.
\evensidemargin 0in        %   Left margin on even-numbered pages.
\textwidth 6.5in
\textheight 9in
\topmargin 0in
\headheight 0in
\headsep 0in

\usepackage[T1]{fontenc}
\usepackage{pslatex}
%\renewcommand*{\familydefault}{\sfdefault}

\usepackage{verbatim}
\newenvironment{qv}%
  {\quote
   \verbatim}%
  {\endverbatim
   \endquote}

\newcommand{\exx}{\\[2ex]}
\newcommand{\code}{\texttt}
\newcommand{\fname}{\texttt}
\newcommand{\eg}{{\em e.g.}}

\parskip 2ex
\parindent 0em


\begin{document}
\sloppypar

\begin{center}
\Large\bf
CIS 201 -- Computer Science I\\
Laboratory Assignment 6\\
\end{center}

\section*{Introduction}

In this lab, you will experiment with creating your own Java classes.

As we have described in lecture and your text, a Java \code{class}
definition is a blueprint for creating objects.  When an object is
created from a \code{class} definition, that object has properties
described by the values of its field variables and provides the
services advertised by its methods.  We think of the properties as
attributes ({\em e.g.}, a FANG \code{Sprite} has a \code{Location2D} and a
\code{Color}) and the methods as verbs
({\em e.g.}, you can \code{translate} and \code{rotate} a FANG \code{Sprite}).

%% A Java class is said to define a {\em type}.

The {\em constructor} of a class is a special method that describes
how to create an {\em instance} ({\em i.e.}, an object) of that class.  A
constructor, called when you use the \code{new} keyword, is used only
once -- when the object is created, whereas an object's other methods
can be used with the object during its entire lifetime.

You have been using constructors for FANG objects since the very first
day of lab. The following line, for example, constructs a new
\code{OvalSprite} and assigns a reference to it to the locally
declared variable \code{round}.
This line is an example of a {\em declaration}
(the stuff before the \verb'=' sign)
and a {\em definition}
(the stuff beginning with \verb'round = new ...').

\begin{qv}
OvalSprite round = new OvalSprite(0.10, 0.10);
^          ^     ^ ^   ^          ^^^^^^^^^^
|          |     | |   |                   \_ parameters to the
|          |     | |   |                      constructor
|          |     | |   \_ Name of constructor to call
|          |     | |      (same as name of class to construct)
|          |     | \_ keyword "new", so Java calls a constructor
|          |     \_ assignment operator (puts the reference to 
|          |         the new object into the box associated with 
|          |         the variable named on left of assignment)
|          \_ name given to the variable that will hold a reference
|             to the object (an OvalSprite)
\_ type of the variable being declared
\end{qv}

To use a Java class defined by FANG
(or anyone else),
you create one or more instances of the class and then
use the instances (the objects) to carry out useful work.
Remember that an object can exist only when it has been created
using the \verb'new' operator.

When you create your own class, you have control over the properties
(field variables)
that objects of that class should have and the services
(methods) that objects of that class should provide.

Many times, you create your own class in a way that \code{extend}s
another class -- as in our \code{NewtonsApple} class that
\code{extend}s the \code{Game} class.  That way we can use the
properties and methods that the other class describes, while adding
our own properties and methods that makes our class special.
As another example, we could have a \code{Building} class
whose properties include number of rooms, number of stories,
and so forth.  A \code{House} class could extend the \code{Building}
class by adding additional properties such as the number of bedrooms.

\section*{How to make a class}

To create a Java class, we first create a Java source file (of the
form \fname{<classname>.java}).
After we have created this source file, we use the Java compiler to
create a Java class file (of the form \fname{<classname>.class}).
The class file consists of a binary representation of the class,
including encoded descriptions of the field variables and methods.
You would not generally want to look at a class file using
an editor, but if you did, you would see the names of your field
variables and methods.

Remember that the Java compiler requires
that the name of the file must match the
name of the class: \code{public class X} must be defined in
\fname{X.java} (and will compile to the class file
\fname{X.class}). Convention (and grading criteria in this class)
require that the names of classes (and thus the files in which they
are defined) begin with capital letters and appear in camel-case. 

Start coding by creating a \fname{Lab06} directory in which you and
your partner will save your work. Note that this time there will be
\emph{multiple} \fname{.java} files in this directory.

First, create a Java source file that defines the class \code{MyName}
by creating (using emacs or your favorite editor)
a file named \code{MyName.java} that has the following
contents:

\begin{qv}
public class MyName {

  private String myName;

  public String getName() {
    return myName;
  }

  public String getXYZZY() {
    return "XYZZY";
  }
}
\end{qv}

Compile this program and verify that your working directory now has a
class file named \code{MyName.class}.

\subsection*{Checkpoint 1}
{\bf
Run this program from within \fname{emacs} or from a shell window.\\
What happens?\\
Show us your results at this point.
}

\section*{Using MyName in a game}
Now start editing a new file named \code{MyNameGame.java}, with the
following contents.
{\em If you are using emacs, use the existing emacs process
to edit this file instead of launching a separate emacs process.}

\begin{qv}
import fang.core.Game;
import fang.sprites.StringSprite;

public class MyNameGame extends Game {

  private MyName mn; // declare a reference to a MyName object
  private StringSprite mnSprite;

  public void setup() {
    mn = new MyName();
    mnSprite = new StringSprite();
    mnSprite.scale(0.05);
    mnSprite.leftJustify();
    mnSprite.setLocation(0.1, 0.5);
    mnSprite.setColor(getColor("white"));
    mnSprite.setText(mn.getName());
    
    addSprite(mnSprite);
  }
}
\end{qv}

Save and compile this program.  You should see no errors (assuming
you've typed in everything correctly).  You should also notice that
you have a class file named \code{MyNameGame.class} in your working
directory.
    
\section*{Checkpoint 2}
{\bf
Run the \code{MyNameGame} program from within \fname{emacs}.\\
What happens?\\
Show us your results.
}

\section*{Null Pointer Exceptions}

As you saw, the Java runtime system encountered a problem when you
tried to run your program.  Notice that your program {\em compiled}
correctly but that it failed to {\em run} correctly, and that at the
top of the (long) list of problems, the Java runtime system said
``Null Pointer Exception''.  This means that it tried to refer to an
object, but that there was no object to refer to.

In this case, the offending part was the value of \code{mn.getName()}.
The \code{getName} method in your \code{MyName} class returns
the value of the \code{myName} variable,
but this variable was only declared and never defined.
Java automatically sets references to \code{null}
when they are not defined.

You can fix this in \code{MyNameGame.java} (temporarily) by replacing
the \code{getName} call with \code{getXYZZY}, which {\em does} return
a value.  Compile and run your program after having made this change,
and verify that it displays \code{XYZZY} on the screen.

Now modify this program by commenting out the line that says

\begin{qv}
mn = new MyName();
\end{qv}

Compile and run your program, and observe that you get another Null
Pointer Exception error.  This time, the error is that \code{mn}
starts off with a value of \code{null}, and when you finally refer to
it in the \code{setText} line, nothing is there.

\subsection*{Checkpoint 2}
{\bf
Show us your work at this point.
}

\section*{Setting your name}

Be sure to un-comment the \code{new MyName()} call before you proceed.

There are two ways to give a value to your \code{myName} field
in the \code{MyName} class.
The first is to create \code{setName} method, as follows:

\begin{qv}
public void setName(String n) {
  myName = n;
}
\end{qv}

Notice that this method has a \code{void} return type, which means
that its only purpose in life is to have a \emph{side-effect}; it does
not determine or return a value.  In this case, the side-effect is to
modify the \code{myName} field.

Add this method to your \code{MyName.java} file, either before or
after the \code{getName} method (you choose), and compile 
\code{MyName.java}.  You should not see any errors.

Now return to editing your \code{MyNameGame.java} file.  In
\code{setup}, just after you create the \code{MyName} object, add the
following line:

\begin{qv}
mn.setName("<your name goes here>");
\end{qv}

Of course, replace \code{"<your name goes here>"} with whatever string
represents \emph{your} name.  Now, change the method call
\code{getXYZZY} back to \code{getName}.

Compile and run your \code{MyNameGame.java} program.

\subsection*{Checkpoint 3}
{\bf
Show us your results at this point.
}

\section*{Constructing your name}

Methods like \code{setName} and \code{getName} are called
(appropriately!) {\em setters} and {\em getters}, respectively.

They provide services to those using object of the class which allow
the user to {\em get} or {\em set} the value of a field variable,
which is (in our case) has \code{private} visibility and can't be
otherwise seen from outside the object.

\emph{Data hiding} or the use of \code{private} data fields is an
example of using abstraction to overcome the complexity of a computer
program.  By making all changes to \code{myName} through methods
defined in \code{MyName}, we can make sure that if we trust the
\code{MyName} class, no other class can violate the integrity of the
fields in the class.

Instead of using a setter to set the \code{myName} field, we can set
this value when we initiall construct the object using the \code{new}
operator.

To do this, we will need to add a new ``method'' to our \code{MyName}
class, called a {\em constructor}.  While a constructor looks like other
methods, it has no return type.  Also a constructor {\em always} uses
the name of the class as its name.  Finally, a constructor will almost
always have a \code{public} visibility.

So in our case, we can define a constructor in our \code{MyName.java}
file as follows:

\begin{qv}
public MyName(String n) {
  myName = n;
}
\end{qv}

Put this code just after the definition of the \code{myName} field at
the beginning of your program.

Compile your \code{MyName.java} program and be sure it doesn't have
any errors.

Now change your ``client'' program \code{MyNameGame.java} by replacing
the lines that say

\begin{qv}
mn = new MyName();
mn.setName("<your name goes here>");
\end{qv}

with the {\em single} line that says

\begin{qv}
mn = new MyName("<your name goes here>");
\end{qv}

(Again, replace the \code{"<your name goes here>"} with a string that has
your name in it.)

Compile you program and run it.

\subsection*{Checkpoint 4}
{\bf
Show us your results at this point.
}

\section*{Religion}
Should you have both a constructor and a setter in your
\code{MyName} class?

If you \emph{never} expect to change the value of \code{myName} after
creating a \code{MyName} object, then setting the value in the
constructor would work best. (If you keep the constructor but remove
the setter, no client program can ever change the value of
\code{myName}.)

If you {\em ever} expected to change the value, then you should keep
the setter, and (I would suggest) remove the constructor.

\textbf{My reason:} its best to have just one way to do something, so
that you'll never have surprises.  But this is a religious issue, not
a Java standard. Your mileage (and that of your instructor) may vary.

\section*{Silly extension}
Now create a new Java file named \code{StrSpr.java}, with the
following code:

\begin{qv}
import fang.sprites.StringSprite;

public class StrSpr extends StringSprite {
}
\end{qv}

Compile this program.  (Don't try to run it...)

You have defined a new class named \code{StrSpr} whose objects have
inherited {\em all} of the properties and methods of
\code{StringSprite}, and has changed/added nothing.

Now return to your \code{MyNameGame.java} program
and make the following changes:

\begin{enumerate}
\item
  comment out the import line
  that refers to \code{fang.sprites.StringSprite}
\item
  change the type of \code{mnSprite}
  from \code{StringSprite} to \code{StrSpr}
\item
  change the line where the \code{StringSprite} constructor is called
  to refer instead to the \code{StrSpr} constructor.
\end{enumerate}

Compile and run your program.

\subsection*{Checkpoint 5}
{\bf
Describe to us what happens.\\
In particular,
explain what the \code{StrSpr} class accomplished that is different from
the \code{StringSprite} class.
}



\section*{Assignment 3}
Begin work on Assignment 3.


\end{document}
