\documentclass[12pt]{article}

\pagestyle{plain}
\oddsidemargin 0in       %   Left margin on odd-numbered pages.
\evensidemargin 0in        %   Left margin on even-numbered pages.
\textwidth 6.5in
\textheight 9in
\topmargin 0in
\headheight 0in
\headsep 0in

\usepackage[T1]{fontenc}
\usepackage{pslatex}
%\renewcommand*{\familydefault}{\sfdefault}

\usepackage{verbatim}
\newenvironment{qv}%
  {\quote
   \verbatim}%
  {\endverbatim
   \endquote}

\newcommand{\exx}{\\[2ex]}
\newcommand{\code}{\texttt}
\newcommand{\fname}{\texttt}
\newcommand{\eg}{{\em e.g.}}

\parskip 2ex
\parindent 0em


\begin{document}
\sloppypar

\begin{center}
\Large\bf
CIS 201 -- Computer Science I\\
Laboratory Assignment 10\\
\end{center}

\section*{Introduction}
In this lab you will manipulate images using classes
from the Georgia Tech MediaComp project.
See \verb'http://coweb.cc.gatech.edu/mediaComp-plan/101'
for more information on this project.

Files from this project can be found in
\begin{qv}
/home/student/Classes/MediaComp
\end{qv}
and the documentation for the Java class used in this project can be found
by following the \verb'Documentation' link on the CS homepage.

\section*{Reading and displaying a picture}
Copy the file
\begin{qv}
/home/student/Classes/201/Labs/Lab10/Pic.java
\end{qv}
into a suitable working directory.
You should be able to compile and run this file directly,
with a copy of the raptor picture displayed on your screen.

File other \verb'.jpg' files on your system
(use the \verb'locate' command!),
or find ones on the web and download them to your current directory.
Modify the program so that the picture to be displayed
is given as a command line parameter, in \verb'args[0]'.

\subsection*{Checkpoint 1}
{\bf
Show us some of your pictures!
}

\section*{Negative images}
Copy the file
\begin{qv}
/home/student/Classes/201/Labs/Lab10/Pic1.java
\end{qv}
into your working directory
and change the \verb'/* FIXME */' part
so that it will use the command line parameter \verb'args[0]'
as the filename.

If you compile and run this program, it should display the picture
exactly as the previous program,
since the ``new'' pixels defined using \verb'setColor'
have the same red, green, and blue values as the original ones.

To make a ``negative image'', you will want
to change the values of red, green, and blue values
so that they go from 255 to zero instead of from zero to 255.
Specifically, given the value \verb'r' for red,
find a simple expression that will evaluate to 255 if \verb'r' is zero
that will evaluate to zero if \verb'r' is 255,
and that will handle values inbetween as well,
as illustrated in the following table
\centerline{%
\begin{tabular}{r|r}
old {\tt r} & new {\tt r}\\ \hline
0 & 255 \\ \hline
1 & 254 \\ \hline
$\cdots$ & $\cdots$ \\ \hline
255 & 0 \\
\end{tabular}
}

Then do the same thing for green and blue.
Your actual parameters
\verb'r', \verb'g', and \verb'b'
in the \verb'new Color(...)' constructor should be replaced
by these expressions.
  
Run your modified program with several pictures of your choosing.

(If you wish, you can use the grayscale conversion as described in class
to produce a black-and-white ``negative'' image
instead of the color one that we are describing here.)

\subsection*{Checkpoint 2}
{\bf
Show us your work at this point.
Your displayed pictures should look like ``negatives''.
}

\section*{Working with pixels in the $(x,y)$ coordinate system}

Instead of treating the image pixels as a one-dimensional array,
we can also look at them in the $(x,y)$ coordinate system,
where (as we do in FANG), the origin $(0,0)$ is at
the top-right corner.
The height and width of the picture can be obtained
using methods defined in the \verb'Picture' class.

Copy the file
\begin{qv}
/home/student/Classes/201/Labs/Lab10/Pic2.java
\end{qv}
into your working directory
and examine it using an editor.
You can see that the program produces a picture
where all of the reds are removed
({\em i.e.}, the red values are all zero).

Use this idea to modify the program
so that the resulting picture mirrors the left-hand side
onto the right-hand side of the picture.
You can do this by scanning only the left half of the picture
(with the $x$ values going from 0 to half of the width)
and changing the color of the corresponding right pixel
using something like
\begin{qv}
rpx.setColor(lpx.getColor());
\end{qv}
where \verb'rpx' is the right pixel
and \verb'lpx' is the left pixel.
(Of course, these aren't the actual variables you'll use.)

\subsection*{Checkpoint 3}
{\bf
Show us your work at this point.
}

\section*{Prisoner}
Make another copy of this program named \verb'Pic3.java'.
Change it so that it puts the image ``behind bars''
by putting black vertical bars strategically
across the image.

To accomplish this, first define a method
with the following documentation and signature:
\begin{qv}
/**
 * draw a black vertical bar on a picture
 * @param p  - the Picture
 * @param bw - the width (in pixels) of the bar to be drawn
 * @param x  - the starting x-coordinate of the bar
 */
public static void drawBar(Picture p, int bw, int x)
\end{qv}

To test your \verb'drawBar' moethod,
make a call to \verb'drawBar' in your \verb'main' method
to see that it works with a simple case.
Then write a \verb'for' loop that will draw several vertical bars,
all of the same width, across your picture.

\subsection*{Checkpoint 4}
{\bf
Show us your work at this point.
}

\section*{Try something!}
Well, try something of your own choosing.
\subsection*{Checkpoint 5}
{\bf
And show us!
}

\end{document}
