\documentclass[12pt,twoside]{memoir}
\usepackage{graphicx}
\usepackage[usenames]{color}
\usepackage{listings}[2000/08/23] 

\usepackage{exercise}

\definecolor{nicered}{rgb}{.647,.129,.149}
\definecolor{listinggray}{gray}{0.9}
\definecolor{templategrey}{gray}{0.30}
\definecolor{commandlinebackground}{gray}{0.1}
\definecolor{commandlineforeground}{gray}{0.9}
\definecolor{commandpromptforeground}{gray}{0.9}

\DeclareGraphicsExtensions{.eps,.pdf,.png,.gif,.jpg}


% ---------- listing code parameters
\lstset{language=java,
  basicstyle=\small\ttfamily,
  numbers=left, 
  numberstyle=\tiny, 
  stepnumber=1, 
  numbersep=5pt,
  frame=single,
  captionpos=b,
  rulecolor=\color{nicered}
}

% ----- Command-line listing language
\lstdefinelanguage{cline}
{
  morecomment=[s][\color{commandpromptforeground}]{\~}{\%},
}

% ----- Command-line environment 
\lstnewenvironment{commandline}[1][]
  {\lstset{language=cline,numbers=none,frame=none,backgroundcolor=\color{commandlinebackground},basicstyle=\color{commandlineforeground},nolol,#1}}
  {}

\newcommand{\artDir}{../art}
\newcommand{\lab}[1]{%
\title{CIS 201 Computer Science I\\Fall 2009 Lab #1}%
\maketitle%
}

\setlength{\hoffset}{0in}
\setlength{\voffset}{0in}
\settypeblocksize{9.5in}{7.5in}{*}
\setlrmargins{0.5in}{*}{1}
\setulmargins{0.75in}{*}{*}
\setheadfoot{\onelineskip}{2\onelineskip}
\checkandfixthelayout
\setcounter{Exercise}{3}
\begin{document}
\lab{01b}

{\color{templategrey}
\begin{itemize}
\item Learn how to log into your Linux account.
\item Learn how to log into your Moodle account.
\end{itemize}
}
\begin{itemize}
\item Learn where class resources reside on the Linux boxes.
\item Learn how to create and navigate directories.
\item Learn how to start (and stop) the Emacs text editor.
\item Learn how to start a \texttt{.java} file.
\item Learn how to compile a \texttt{.java} file to produce a
  \texttt{.class} file.
\item Learn how to run a \texttt{.java} (or rather, \texttt{.class}) file.
\item Learn how to use a Java class file as a \emph{template}.
\item Learn how to correct \emph{compiler errors}.
\end{itemize}

\begin{Exercise}[name={Checkpoint},title={Launch a terminal.}]
  Linux provides a desktop environment similar to that found in other
  operating systems (Windows XP/Vista or Mac OS X). It also offers a
  \emph{command-line interface}, or a \emph{shell} where the user can
  type commands into a command interpreter which runs the commands and
  then prompts the user for a new command. This class uses the
  command-line interface a great deal.

  A shell runs inside of a terminal; go to the menu at the upper-left
  of the desktop and find \textbf{Accessories $|$ Terminal}. Launching
  it should give you a window that has a prompt something like this:


\begin{commandline}
 laddbc@haystack:~$  _
\end{commandline}
% $

The bit before the \texttt{\@} is your login name (Dr. Ladd's is
\texttt{laddbc}). Between the \texttt{\@} and the \texttt{:} is the name 
of the machine you are logged into. Between the
\texttt{:} and the \texttt{\$} is the \emph{current directory path}. 
Like many computer file systems, the Linux file system uses
 hierarchical \emph{directories} or \emph{folders}. Entries in the 
file system are either files, containing information (\emph{e.g.}, a
Java program, a term paper, your favorite MP3) or a directory, a
container for files and folders. For historic reasons, Linux
refers to these as directories; folder and directory are used
interchangeably in this class.

Course files: The directory \texttt{/home/students/Classes/201} is the
home directory for this class \emph{on the lab computers}. The same
materials and sample files you can find through the Web (Moodle links
go into this space) reside here. Lecture notes (sample programs), lab
assignments, all are here. Another way to have gotten \emph{this}
document would have been to copy
\texttt{/home/students/Classes/201/Labs/Lab01/lab01b.pdf} to your home
directory (or just open it with the a PDF reader such as
\texttt{evince}).

The directory path \texttt{\~} is special: it is the home directory of
the currently logged in user. So, the above prompt is saying that
Dr. Ladd is in his own home directory. This is where the terminal
program starts. It is the folder where you save all of your work (or
rather, where you put subdirectories to save your work). 

\emph{Your} home directory follows you when you log onto any lab
machine. On \texttt{haystack} or \texttt{algonquin} or any
other lab machine \emph{your} home directory is automatically mounted
when you sign on.

Note that the \texttt{\_} after the prompt represents where you can
type. It is not shown in other examples.
\end{Exercise}
\noindent
\textbf{Show your work on Checkpoint~\theExercise{} to the lab monitor;
  have them sign off on it before continuing.}

\begin{Exercise}[name={Checkpoint},title={Create a CS1 and a Lab01 folder.}]
To create a folder, you can use the \texttt{mkdir} (make directory)
command with the name of the new directory. To change the current
directory, you can use the \texttt{cd} (change directory) command. To
create a \texttt{CS1} directory and then, in that directory create a
\texttt{Lab01} directory, the commands are:

\begin{commandline}
 laddbc@haystack:~$ mkdir CS1
 laddbc@haystack:~$ cd CS1
 laddbc@haystack:~/CS1$ mkdir Lab01
 laddbc@haystack:~/CS1$ cd Lab01
 laddbc@haystack:~/CS1/Lab01$ ls -a
 . ..
 laddbc@haystack:~/CS1/Lab01$ 
\end{commandline}

\noindent
Notice how the current directory changes as you navigate into
directories below the home directory. Also, notice that in the newly
created directory \texttt{Lab01}, when all of the files are listed
(\texttt{ls} for listing; \texttt{-a} to tell the listing program to list
all), there are already two items: \texttt{.} and \texttt{..}. These are
the current directory (any directory refers to itself with a single
dot) and the parent of the current directory (any directory refers to
the directory it is contained in with two dots). 

How would you change the current directory back to \texttt{~/CS1}?
\end{Exercise}
\noindent
\textbf{Show your work on Checkpoint~\theExercise{} to the lab monitor
  and answer the question above for them.  Have them sign off on it
  before continuing.}

\begin{Exercise}[name={Checkpoint},title={Starting and stopping \texttt{emacs}}]
CIS 201 uses the \texttt{emacs} text editor. A \emph{text
  editor} is a lot like a word processor. The primary difference is
that while the displayed text may have different colors and different
fonts, the text editor saves \emph{just the characters}
and not their formatting. This is important because the Java compiler
doe not understand bold face or italics.

To run \texttt{emacs}, you can either type the name of the program into
the shell \emph{or} you can select \texttt{emacs} from the applications
menu on the desktop. When you type the name in, you can also specify
the name of a file you wish to edit. Let us assume that we want to
edit a file in \texttt{Lab01} called \texttt{NewtonsApple.java} (notice the
\texttt{.java} at the end; the Java compiler requires that if we are
going to compile our Java code). We can just type: 

\begin{commandline}
 laddbc@haystack:~/CS1/Lab01$ emacs NewtonsApple.java &
 [1] 1234
 laddbc@haystack:~/CS1/Lab01$ 
\end{commandline}
%$
\noindent
What is that \texttt{\&} doing there? It is a special character telling
the shell to run the command given but not to wait until it is
finished. Before, when we listed the files, the next shell prompt
didn't appear until the command was finished. Here we have the next
shell prompt \emph{and} \texttt{emacs} is running. The \texttt{[1] 1234}
is an indication that we are running one command in the background and
what \emph{process id} the operating system gave \texttt{emacs};
ignore that number for the moment.

If you look at the top of the \texttt{emacs} window, there is a standard
menu bar. If you look in the \textbf{File} menu, the
last item is \textbf{Exit Emacs}. Also in that directory are the
\textbf{Save File} and \textbf{Save File As} commands. Those commands
have their keyboard equivalents written next to them in the menu;
\textbf{C-} in front of a character means press the \textbf{Ctrl} key
while pressing the key.

Notice in the top line menu that there is also an entry for
\textbf{JDE}. This is short for Java Development Environment, an
\texttt{emacs} extension that makes it a little easier to write and
compile Java.

It is time to write some Java code. Type the following into your
\texttt{emacs} window and save it:

\begin{lstlisting}
import fang2.core.Game;

public class NewtonsApple
    extends Game {
}
\end{lstlisting}

Having saved the file, you can now compile it. You can select
\textbf{JDE $|$ Compile}. The blank line at the bottom of the
\texttt{emacs} window (known as the \emph{minibuffer} in
\texttt{emacs} parlance) says something about starting the
``Beanshell'', then the main window splits and a window labeled
\emph{*JDEE Compile Server*} appears with the output of the
compilation. Upon success the output looks like this
(\texttt{-classpath} line broken across 3 line for readability):

\begin{lstlisting}[numbers=none]
CompileServer output:

-classpath /home/faculty/laddbc/CS1/Lab01:
           /usr/local/share/java/jdk1.6.0/lib/fang2.jar 
           /home/faculty/laddbc/CS1/Lab01/NewtonsApple.java

Compilation finished at Sat Aug 23 16:37:57
\end{lstlisting}

To get rid of the split in the window you can, with the cursor in the
top window, select \textbf{File $|$ Unsplit Window} (or use the
keyboard shortcut of \textbf{C-x 1} to get just one (1) window). 

Go over to your shell window and run \texttt{ls -a} again:

\begin{commandline}
 laddbc@haystack:~/CS1/Lab01$ ls -a
 . .. NewtonsApple.class NewtonsApple.java semantic.cache semantic.cache~
\end{commandline}
% $

\noindent
The two \texttt{NewtonsApple} files are the compiled file
(\texttt{.class}) and the Java source file (\texttt{.java}, the one we
edited and saved from \texttt{emacs}). Ignore the \texttt{semantic.cache}
file (created by \texttt{emacs} when it edits source
code). 
\end{Exercise}
\noindent
\textbf{Show your work on Checkpoint~\theExercise{} to the lab monitor;
  have them sign off on it before continuing.}

\begin{Exercise}[name={Checkpoint},title={Run the program}]
  Now, run the program. That is, in \texttt{emacs}, after the compile
  step succeeds, run the program by selecting \textbf{JDE $|$ Run
    App}. The window splits again and a new window appears labeled
  \emph{NewtonsApple} with several buttons across the bottom. Nothing
  else happens at this point, so just press the \textbf{Quit} button.
\end{Exercise}
\noindent
\textbf{Show your work on Checkpoint~\theExercise{} to the lab monitor;
  have them sign off on it before continuing.}
\newpage
\begin{Exercise}[name={Checkpoint},title={Continue the program.}]
Now, modify your program by updating the contents of
\texttt{NewtonsApple.java} to be: 

\begin{lstlisting}
import fang2.core.Game;
import fang2.sprites.OvalSprite;
import fang2.sprites.RectangleSprite;

public class NewtonsApple extends Game {
  private OvalSprite apple;
  private RectangleSprite newton;

  public void setup() {
    apple = new OvalSprite(0.10, 0.10);
    apple.setColor(getColor("red"));
    dropApple();

    newton = new RectangleSprite(0.10, 0.10);
    newton.setColor(getColor("green"));
    newton.setLocation(0.5, 0.9);

    addSprite(apple);
    addSprite(newton);
  }

  public void dropApple() {
    apple.setLocation(random.nextDouble(), 0.0);
  }
}  
\end{lstlisting}

Now when you save, compile, and run the program, an apple appears,
just off the top of the screen, and a green Newton is visible below.
\textbf{Start} starts the game but the game does nothing; there is no
method to advance the game.

\textbf{Correcting Typos}: If you have any compiler errors,
\texttt{emacs} shows them to you. Consider, for example, misspelling
\texttt{Color} as \texttt{Colour}. Then the CompileServer  output (in the
bottom window) would look like this:

\begin{lstlisting}[numbers=none]
CompileServer output:

-classpath /home/faculty/laddbc/CS1/Lab01:
           /usr/local/share/java/jdk1.6.0/lib/fang2.jar 
           /home/faculty/laddbc/CS1/Lab01/NewtonsApple.java

/home/faculty/laddbc/CS1/Lab01/NewtonsApple.java:11: cannot find symbol
symbol  : method setColour(java.awt.Color)
location: class fang2.sprites.OvalSprite
    apple.setColour(getColor("red"));
         ^
/home/faculty/laddbc/CS1/Lab01/NewtonsApple.java:15: cannot find symbol
symbol  : method setColour(java.awt.Color)
location: class fang2.sprites.RectangleSprite
    newton.setColour(getColor("green"));
          ^
2 errors

Compilation exited abnormally with code 1 at Mon Aug 25 15:48:00  
\end{lstlisting}

\texttt{emacs} moves the cursor down to the first
\texttt{/home/faculty/...} line, and also move the cursor in the
\texttt{NewtonsApple.java} buffer to line 11 (the line with the
error). The first line in the error window identifies the file and
line with the error and the error the Java compiler encountered. In
this case the error is \texttt{cannot find symbol}. The next line gives
the symbol that could not be found (\texttt{setColour}) and the line
after that gives the class where the Java compiler expected to find
the symbol (\texttt{OvalSprite}). 

The next two lines are the offending line from the \texttt{.java} file
and a line with a \texttt{\^} pointing to the spot where the Java
compiler realized there was an error. In this case, we see that the
method name in our listing was \texttt{setColor} though we typed
\texttt{setColour}. Removing the ``u'' fixes the problem.

After correcting one error you can move to the next error by
pressing \texttt{C-x `} (that is a back-tick, next to the \texttt{1} on
most keyboards).  The error message window and the source code window
both move to the next error.
\end{Exercise}
\noindent
\textbf{Show your work on Checkpoint~\theExercise{} to the lab monitor;
  have them sign off on it before continuing.}

\begin{Exercise}[name={Checkpoint},title={Finish \texttt{NewtonsApple}.}]
Here's the finished listing for Newton's Apple. 

\begin{lstlisting}
import fang2.attributes.Location2D;
import fang2.core.Game;
import fang2.sprites.OvalSprite;
import fang2.sprites.RectangleSprite;
import fang2.sprites.StringSprite;

public class NewtonsApple extends Game {
  private OvalSprite apple;
  private RectangleSprite newton;
  private int applesCaught;
  private int applesDropped;
  private StringSprite displayScore;

  public void setup() {
    applesCaught = 0;
    applesDropped = 0;

    displayScore = new StringSprite(); // no text to display yet
    displayScore.scale(0.10);
    displayScore.setColor(getColor("white"));
    updateScore(); // updated the content of the displayed score

    apple = new OvalSprite(0.10, 0.10);
    apple.setColor(getColor("red"));
    dropApple();

    newton = new RectangleSprite(0.10, 0.10);
    newton.setColor(getColor("green"));
    newton.setLocation(0.5, 0.9);

    addSprite(apple);
    addSprite(newton);
    addSprite(displayScore);
  }

  public void advance(double secondsSinceLastCall) {
    Location2D position = getPlayer().getMouse().getLocation();
    if (position != null) {
      newton.setX(position.x);
    }

    apple.translateY(0.33 * secondsSinceLastCall);

    if (apple.intersects(newton)) {
      applesCaught = applesCaught + 1; // another apple caught
      updateScore();
      dropApple();
    }

    if (apple.getY() >= 1.0) {
      updateScore();
      dropApple();
    }
  }

  public void dropApple() {
    apple.setLocation(random.nextDouble(), 0.0);
    applesDropped = applesDropped + 1; // another apple dropped
  }

  public void updateScore() {
    displayScore.setText("Score: " + applesCaught + "/" + applesDropped);
    displayScore.setLocation(displayScore.getWidth() / 2, 
                             displayScore.getHeight() / 2);
  }
}  
\end{lstlisting}

Modify your program, save, compile, and run \texttt{NewtonsApple}.
Save your work, compile it, correct any compilation errors, and run
it. You should have a working game running.
\end{Exercise}
\noindent
\textbf{Show your work on Checkpoint~\theExercise{} to the lab monitor;
  have them sign off on it before continuing.}

\Large{Log off of the lab computer you are using before leaving they
  lab. Anyone entering the lab has unlimited access to your files if
  you remain logged on. \textbf{DO NOT} turn off lab computers! They
  are a shared resource and there might be someone else logged in to
  ``your'' machine.}

\end{document}

%%% Local Variables: 
%%% mode: latex
%%% TeX-master: t
%%% End: 

% LocalWords:  Moodle Ladd's login
