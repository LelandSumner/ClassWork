\documentclass[12pt,twoside]{memoir}
\usepackage{graphicx}
\usepackage[usenames]{color}
\usepackage{listings}[2000/08/23] 

\usepackage{exercise}

\definecolor{nicered}{rgb}{.647,.129,.149}
\definecolor{listinggray}{gray}{0.9}
\definecolor{templategrey}{gray}{0.30}
\definecolor{commandlinebackground}{gray}{0.25}
\definecolor{commandlineforeground}{gray}{0.85}
\definecolor{commandpromptforeground}{gray}{0.55}


\DeclareGraphicsExtensions{.eps,.pdf,.png,.gif,.jpg}


% ---------- listing code parameters
\lstset{language=java,
  basicstyle=\small\ttfamily,
  numbers=left, 
  numberstyle=\tiny, 
  stepnumber=1, 
  numbersep=5pt,
  frame=single,
  captionpos=b,
  rulecolor=\color{nicered}
}

% ----- Command-line listing language
\lstdefinelanguage{cline}
{
  morecomment=[s][\color{commandpromptforeground}]{\~}{\%},
}

% ----- Command-line environment 
\lstnewenvironment{commandline}[1][]
  {\lstset{language=cline,numbers=none,frame=none,backgroundcolor=\color{commandlinebackground},basicstyle=\color{commandlineforeground},nolol,#1}}
  {}

\newcommand{\artDir}{../art}
\newcommand{\lab}[1]{%
\title{CIS 201 Computer Science I\\Fall 2009 Lab #1}%
\maketitle%
}

\setlength{\hoffset}{0in}
\setlength{\voffset}{0in}
\settypeblocksize{9.5in}{7.5in}{*}
\setlrmargins{0.5in}{*}{1}
\setulmargins{0.75in}{*}{*}
\setheadfoot{\onelineskip}{2\onelineskip}
\checkandfixthelayout

\begin{document}

\lab{01a}

\begin{itemize}
\item Learn to log into your Linux account.
\item Learn to log into your Moodle account.
\end{itemize}
{\color{templategrey}
\begin{itemize}
\item Learn where class resources reside on the Linux boxes.
\item Learn how to create and navigate directories.
\item Learn how to start (and stop) the Emacs text editor.
\item Learn how to start a \texttt{.java} file.
\item Learn how to compile a \texttt{.java} file to produce a
  \texttt{.class} file.
\item Learn how to run a \texttt{.java} (or rather, \texttt{.class}) file.
\item Learn how to use a Java class file as a \emph{template}.
\item Learn how to correct \emph{compiler errors}.
\end{itemize}
}

\begin{Exercise}[name={Checkpoint},title={Log into your Linux account.}]

  All labs in CIS 201 are to be done in pairs. One student should log
  on and create the appropriate directories and files in their
  computer space. When the lab is done, the other partner should get a
  copy of the files.

  Find your partner: each week are assigned a partner selected
  at random from the rest of the class (avoiding duplication of
  previous partners if possible). The two of you should sit down in
  front of one machine in Dunn 358 (\emph{aka} the ``Linux Lab'', the
  ``CS Lab'', or just the ``Lab''). You may have to press a key on the
  keyboard or move the mouse to ``wake up'' the machine.

  \textbf{You should never turn off the computers in the lab.} They
  are a departmental resource and others could be logged in to any
  given machine remotely. You should never \emph{need} to turn a lab
  machine on.

  When the screen comes on there is a login prompt. Type in your
  \emph{campus computing account} username and password (the password
  types as dots to prevent anyone reading it over your
  shoulder). Press \texttt{Enter} or click on the \texttt{Login} button.

  You are now logged into the machine and see the \emph{XWindows
    Desktop}. It is similar to other operating system's desktop: there
  are windows, icons, a menus, and a pointer\footnote{a standard
    W.I.M.P. interface}. The menu in the upper left corner contains
  applications that you can run including the \textbf{Accessory $|$
    Terminal} and the \emph{Firefox} Web browser.
\end{Exercise}
\noindent
\textbf{Show your work on Checkpoint~\theExercise{} to the lab monitor;
  have them sign off on it before continuing.}

\begin{Exercise}[name={Checkpoint},title={Log into Moodle}]

  Moodle is a Web-based classroom management system. What that means
  is that slides, schedules, notes, resources, grades, and even labs
  for this course (or any CS course, for that matter) can be stored on
  Moodle. Navigate on the Web to the CS Department's Moodle pages,
  to this class's Moodle site, create an account, log into
  the class, and retrieve and print the remainder of this lab.

  The Firefox Web browser is installed on the Linux machines. You can
  find it in the \textbf{Applications} entry in the menu. Launch the
  Web browser.

  The Web address of Moodle is:
  \texttt{http://db.cs.potsdam.edu/moodle/}. On that page are
  find entries for several courses. You want to find \emph{CIS~201}
  (it should be the first \textbf{Available Course}). Follow the link
  to \emph{CIS~201}.

  You need to create a \emph{New Account} on Moodle (button to the
  lower right). Provide your \emph{campus computing account} user name
  and your \emph{campus e-mail address} as your e-mail address. The
  partner who is not logged in must also create an account on Moodle;
  there is an assignment posted on Moodle due next week.

  Moodle sends a registration e-mail to the e-mail address
  provided. Follow the directions in it to finish creating your
  account.

  When you have an account and click on the course link, Moodle
  prompts you for an \emph{Enrollment Key}. The key keeps random
  people from reading our course content (and hopefully keeps spammers
  out as well). The key for this course is \texttt{FANG} (one word, no
  spaces). You should only be prompted for the key once per course at
  the beginning of the semester.

  The course homepage on Moodle has the  \textbf{Weekly
    Outline} with the course schedule, assignments, readings, and
  other resources by week. In the first week there are two links,
  \textbf{Lab01a} (points to a \texttt{.pdf} version of this document)
  and \textbf{Lab01b}. Download the \texttt{.pdf} file of the second
  half of the lab and continue from there. 

  If you go to the first week (, you will see a
  link labeled \textbf{Lab 1b}. Download that link and continue your
  lab from there.
\end{Exercise}
\noindent
\textbf{Show your work on Checkpoint~\theExercise{} to the lab monitor;
  have them sign off on it before continuing.}

\end{document}

%%% Local Variables: 
%%% mode: latex
%%% TeX-master: t
%%% End: 

% LocalWords:  Moodle Ladd's login
