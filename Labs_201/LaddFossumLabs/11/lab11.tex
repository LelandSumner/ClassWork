% book.tex -- Copyright (c) 2004, Brian C. Ladd
% 
% Author : Brian C. Ladd 
% Created: Wed Jul 07 12:03:49 2004 by blad
% Revised: Fri Dec 28 15:27:43 2007 by blad
% 
\def\FileCreated{Wed Jul 07 12:03:49 2004}
\def\FileRevised{Fri Dec 28 15:27:43 2007}
\documentclass[12pt,oneside]{memoir}
\usepackage{epsfig}
\usepackage{graphicx}
\usepackage[usenames]{color}
\usepackage{xltxtra}
\usepackage{fontspec}
\usepackage{textcomp}
\usepackage{pstricks}
\usepackage{listings}[2000/08/23] 

\usepackage{rotating}
\usepackage{exercise}
\definecolor{nicered}{rgb}{.647,.129,.149}
\definecolor{listinggray}{gray}{0.9}
\definecolor{templategrey}{gray}{0.30}
\definecolor{commandlinebackground}{gray}{0.25}
\definecolor{commandlineforeground}{gray}{0.85}
\definecolor{commandpromptforeground}{gray}{0.55}
% ----- Fonts -----
\defaultfontfeatures{Scale=MatchLowercase,Mapping=tex-text}
\setmainfont{Gentium Book Basic}
\setmonofont{Inconsolata}
\setsansfont{Arial}
% ----- Fonts -----
\newcommand\code[1]{\lstinline^#1^}
\newcommand\fname[1]{\texttt{#1}}

\lstset{language=java,
  basicstyle=\small\ttfamily,
  numbers=none, 
  numberstyle=\tiny, 
  stepnumber=1, 
  numbersep=5pt,
  frame=single,
  captionpos=b,
  rulecolor=\color{nicered}
}

\lstdefinelanguage{cline}
{
  morecomment=[s][\color{commandpromptforeground}]{ }{ },
} 

\lstnewenvironment{commandline}[1][]
{\lstset{language=cline,numbers=none,frame=none,backgroundcolor=\color{commandlinebackground},basicstyle=\color{commandlineforeground},nolol,#1}}
{}

\newenvironment{Checkpoint}[1]{%
  \begin{Exercise}[name={Checkpoint},title={#1}]}{%
  \end{Exercise}%
  \textbf{Show your work on Checkpoint~\theExercise{} to the lab monitor, %
    answering any necessary questions for them.  Have them sign before continuing.}}

\newcommand{\artDir}{../art}
\newcommand{\lab}[2]{%
  \title{CIS 201 Computer Science I\\Fall 2009 Lab #1\\#2}%
  \maketitle%
}

\setlength{\hoffset}{0in}
\setlength{\voffset}{0in}
\settypeblocksize{9.5in}{7.5in}{*}
\setlrmargins{0.5in}{*}{1}
\setulmargins{0.75in}{*}{*}
\setheadfoot{\onelineskip}{2\onelineskip}
\checkandfixthelayout

\begin{document}

\lab{11}{Reading Files}

\begin{itemize}
\item Working with files (for input and output).
\item Working with \code{String}s.
\item Different kinds of loops.
\end{itemize}

In this lab you will write three increasingly difficult programs: the
first prompts the user for the name of a file and displays the text
file on the screen; the second prompts the user for two file names,
one for input and one for output and copies the contents of the input
file into the output file; the final program copies a file in a
similar way but before writing a \code{String} to the output file, the
string is \emph{enciphered} so that the contents of the copied file
are secure from prying eyes.\footnote{For a sufficiently narrow
  definition of ``prying eyes.''}

\begin{Checkpoint}{ViewFile.java}
Write a non-FANG program, \code{ViewFile.java}. Put the file in an
appropriately named folder and so on. 

The program should
\begin{itemize}
\item Prompt the user for an input file name and read in the file name
  from standard input. (Consider that you must prompt the user for
  multiple file names soon; how can you keep from repeating yourself?)
\item Open the file for input.
\item If there was a problem opening the file for input, provide a
  useful error message to the user. Be sure to include a description
  of the problem and the name of the offending file. Terminate the
  program after printing the error message.
\item If the file opened properly, copy the lines of the file to
  standard output.
\item You can test your program by viewing the \fname{.java} file of
  the program itself.
\end{itemize}
\end{Checkpoint}

\begin{Checkpoint}{CopyFile.java}
Starting with a copy of \fname{ViewFile.java}, change the name to
\code{CopyFile} and make it compile.

\code{System.out} is a \code{PrintStream}. You want to construct a new
\code{PrintStream} associated with an output file. It is similar to
associating a \code{Scanner} with an input file \textbf{except} that
the \code{PrintStream} constructor takes the name of a file (not a
\code{File} object) and you \emph{must} \code{close} the file when you
are done with it. The following loop gives an example of using a
\code{PrintStream} for output.

\begin{lstlisting}
  String fname = /* get the file name somehow */;
  PrintStream ps = null;
  try {
     ps = new PrintStream(fname);
     ps.println("Hello, World!");
  } catch (FileNotFoundException fnf) {
    System.err.println("There was a problem with " + fname);
    System.exit(1);
  } finally {
    ps.close();
  } 
\end{lstlisting}

Your modified program should 
\begin{itemize}
\item Prompt the user for an input file name and read in the file name
  from standard input. (Consider that you must prompt the user for
  multiple file names soon; how can you keep from repeating yourself?)
\item Open the file for input.
\item If there was a problem opening the file for input, provide a
  useful error message to the user. Be sure to include a description
  of the problem and the name of the offending file. Terminate the
  program after printing the error message.
\item Prompt the user for an output file name and read the  file name
  from standard input.
\item Open the file for output.
\item If there was a problem, give a reasonable error message and
  terminate the program.
\item If both files opened properly, copy the lines of the input file to
  the output file.
\item You can test your program by viewing the \fname{.java} file of
  the program itself.
\end{itemize}
\end{Checkpoint}

\begin{Checkpoint}{EncipherFile.java}
Starting with \fname{CopyFile.java}, you should rename it to
\code{EncipherFile.java}. There will be no visible change from the
user's viewpoint: the program will prompt for two file names, one for
input, one for output. It then reads the lines of the input file,
enciphers each line character by character, and then prints out the
enciphered line.

The guts of the program looks like this
\begin{lstlisting}
while (there are more lines) {
  String original = nextLine;
  String converted = "";
  while (there are characters  in original) {
    char nextCh = nextChar;
    add nextCh to end of converted;
  }
  print out converted;
}
\end{lstlisting}

How do you loop across all the characters in a \code{String}? 

It is a count-controlled loop based on the \code{length()} method of
\code{String} and \code{charAt(i)} (look these methods up in the
\code{String} documentation). 

Modify \code{EncipherFile} so that its main loop converts each line
before printing. The following loop does the real work:

  \begin{lstlisting}
// inside the hasNextLine loop...
String converted = "";
for (int characterNdx = 0; 
     characterNdx < orginal.length(); 
     ++characterNdx) {    
  char nextCh = orginal.charAt(characterNdx);
  // this is where you convert nextCh...
  converted = converted + Character.toString(nextCh);
}
// now print out converted
  \end{lstlisting}

You should modify this code so that every semicolon (\code{;}) is
replaced by a caret (\code{\^}). 

You can test the program by running it on its own code. You can then
open the copy and see if the semicolons are all replaced with the
right character (and that the rest of lines are unchanged). 
\end{Checkpoint}

We leave the program here; it is possible that next week you will
extend the enciphering of the characters so that it uses the Caesar
cipher to encipher the text read from the file. It is also possible
that we need to work some more on sorting so that may be next week's
lab. 

\Large{Log off of the lab computer you are using before leaving they
  lab. Anyone entering the lab has unlimited access to your files if
  you remain logged on. \textbf{DO NOT} turn off lab computers! They
  are a shared resource and there might be someone else logged in to
  ``your'' machine.}
\end{document}

%%% Local Variables: 
%%% mode: latex
%%% End: 

% LocalWords:  Moodle Ladd's login emacs
