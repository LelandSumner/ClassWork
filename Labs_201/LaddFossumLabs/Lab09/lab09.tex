\documentclass[12pt]{article}

\pagestyle{plain}
\oddsidemargin 0in       %   Left margin on odd-numbered pages.
\evensidemargin 0in        %   Left margin on even-numbered pages.
\textwidth 6.5in
\textheight 9in
\topmargin 0in
\headheight 0in
\headsep 0in

\usepackage[T1]{fontenc}
\usepackage{pslatex}
%\renewcommand*{\familydefault}{\sfdefault}

\usepackage{verbatim}
\newenvironment{qv}%
  {\quote
   \verbatim}%
  {\endverbatim
   \endquote}

\newcommand{\exx}{\\[2ex]}
\newcommand{\code}{\texttt}
\newcommand{\fname}{\texttt}
\newcommand{\eg}{{\em e.g.}}

\parskip 2ex
\parindent 0em


\begin{document}
\sloppypar

\begin{center}
\Large\bf
CIS 201 -- Computer Science I\\
Laboratory Assignment 9\\
\end{center}

\section*{Introduction}
In this lab you will create a game that reads its configuration
from a file.

\section*{Reading a file}

Consider the process of writing a pinball game. Given a screen and a
ball that bounces, different levels or pinball games frequently differ
in how their bumpers (and other things) are laid out. 

What we really want, then, is to build a pinball \emph{engine}, a
program that can simulate any pinball machine described in an
appropriate way. You will begin with a simulator that has ball physics
but no balls or bumpers. Your job is to extend the program so that it prompts
the user for the name of a file (format described below) and
then creates the appropriate sprites in the machine before setting the
ball(s) loose.

Copy the files \fname{Pinball.java} and
\code{PinballGame.java} into a directory \fname{Lab09}.
You can find the files to copy in the directory
\begin{qv}
/home/student/Classes/201/Labs/Lab09
\end{qv}

Also in that directory is a file named \fname{first.pnd}.
We will be using the extension \fname{.pnd}
to stand for a ``Pinball Description'' file,
an extension we coined.
This file will have a format described later.

Download the Java and pinball files into your lab directory. Compile
and run the pinball game. It should compile but it should not do
anything because there is no description of a pinball game. 

Look at the \code{PinballGame} class. Find all of the
stub functions (they are the ones with little or no code in them).
In this lab, you will replace the stubs with working code.

Modify \code{getFileName} so that it displays the prompt
that is passed into it followed by a single space and then reads a
string from the user, returning that value.

Reading information from the user's keyboard
is somewhat like printing information to the user's screen.
Code such as \verb'System.out.print(...)' will display
the expression to your screen
based on the type of the expression --
such as a \verb'String', an \verb'int', or whatever.
Reading information from your keyboard uses \verb'System.in',
which should not be surprising.
But how will the program know the type of the value to be read?
Here is where the notion of a \verb'Scanner' comes in.

A \verb'Scanner' is a Java object that takes raw input
from the keyboard (or from a file, as we shall see)
and interprets it as a string or integer or whatever,
depending on what the program asks it to do.

How does one read a string from the keyboard, for example?
Here is a snippet of code that would do so:
\begin{qv}
Scanner kbd = new Scanner(System.in); // tokenize raw input from the keyboard
String str = kbd.next(); // interpret the next token from the keyboard
\end{qv}
For a \verb'Scanner' object, the term ``token'' means
a string of input characters that does not have any whitespace
(such as spaces, tabs, or newlines).
The \verb'next()' method skips any whitespace
until it encounters a non-whitespace character
and then collects all of the following non-whitespace characters
into a single string that it returns to the caller.

A \verb'Scanner' can also return \verb'int's using \verb'nextInt()'
and \verb'double's using \verb'nextDouble()', for example.
You should look at the \verb'Scanner' class documentation to see
all of the things that a \verb'Scanner' can process.

Use a \verb'Scanner' as described above
to implement the \verb'getFileName' method.

\subsection*{Checkpoint 1}
{\bf
Compile and run your program and show us your results.
}

\section*{The {\tt ensureExtension} method}

The \code{ensureExtension} method is passed a file name
and returns that file name with a given extension.
For example, if \code{ensureExtension} were passed the parameters
\code{something} and \code{txt}, it would return the string
\code{something.txt}.
The method is called \emph{ensure}Extension
because it should only add the extension if the file name does not
already end with it.
Thus calling the method with parameters \code{something.txt}
and \code{txt} should also return \code{something.txt}.

Before you implement the \code{ensureExtension} method,
look at the documentation for the \code{String} class
and study the following methods:
\begin{qv}
equals
endsWith
\end{qv}

The \code{ensureExtension} method takes two \code{String} parameters.
The first parameter is the file name
(which you get using the \verb'getFileName' method),
and the second is the desired extension.
If the file name does not end with a dot and the given
extension, it needs fixing; otherwise return it unchanged.
To fix a file name, if it ends with a dot, just return file name plus
the extension; otherwise return file name plus dot plus extension.
Here are possible inputs and their corresponding return values:

\centerline{
\begin{tabular}{ll|l}
{\bf fname} & {\bf ext} & {\bf return value} \\ \hline
{\tt abcde} & {\tt pnd} & {\tt abcde.pnd} \\ \hline
{\tt abcde.} & {\tt pnd} & {\tt abcde.pnd} \\ \hline
{\tt abcde.x} & {\tt pnd} & {\tt abcde.x.pnd} \\ \hline
{\tt abcde.pnd} & {\tt pnd} & {\tt abcde.pnd} \\
\end{tabular}
}

You should use the \code{ensureExtension} method
in the \code{readPNDFile} method,
after you get the file name (from \code{getFileName} above) but before
passing it to \code{connectScannerToFile}:
\begin{qv}
fname = ensureExtension(fname, "pnd");
\end{qv}

\subsection*{Checkpoint 2}
{\bf
Show us your working code.
}

\section*{Reading the configuration file}

A PND file is a text file containing
any number of lines.
Each line in a PND file has one of the following formats:

\begin{qv}
Ball <color> <x> <y> <w> <h> <dX> <dY>
Oval <color> <x> <y> <w> <h>
Rectangle <color> <x> <y> <w> <h>
\end{qv}

Here \verb'<color>' is a color name
(such as \verb'red' or \verb'yellow', for example)
that is recognized by the \verb'getColor' method.
The other terms (\verb'<x>', \verb'<y>', etc.)
are decimal numbers (such as \verb'0.2')
that can be interpreted as \verb'double's in Java.

Now you are prepared to read the whole configuration file
by implementing the rest of the \verb'readPNDFile' method.
You will write a loop that
reads the next token from the file
(a token is, in this case, a word)
and processes it according to the structure
of a Pinball Description file as described above.
The \verb'connectScannerToFile' method,
which is provided to you,
will associate the \verb'Scanner' object
with the filename you give when you run the program --
the one with the \verb'pnd' extension.

Reading from a \verb'Scanner' associated with a file
uses a loop that normally looks something like this,
where \verb'in' is assumed to be a \verb'Scanner' object
(your mileage may vary):
\begin{qv}
while (in.hasNext()) {
    String obj = in.next(); // get the next token string
    ...
}
\end{qv}
This loop will continue to read tokens from the file
until there are no more tokens to read --
in other words, until all the contents
of the file have been read and processed.
Here the term ``token'' refers
to character strings that consist entirely
of non-whitespace characters.

If the token that the \verb'Scanner' sees
is one of \verb'Ball', \verb'Oval', or \verb'Rectangle',
you should call a routine that reads the rest of the information for that
type of sprite and creates and adds the sprite to the game canvas.
For a \verb'Ball', you will call \verb'handlePinball';
for a \verb'Oval', you will call \verb'handleOval'; and
for a \verb'Rectangle', you will call \verb'handleRectangle'.
If the token is none of these,
your program should terminate gracefully
with an informative message indicating
that the PND file has the wrong format.

When using a \verb'Scanner',
if you know the type of the next token to be read,
you can simply call the appropriate \verb'Scanner' method such as
\verb'nextInt()' (for an \verb'int' token),
\verb'nextDouble()' (for a \verb'double' token),
or just \verb'next()' (for a \verb'String' token).
For example, if you have a PND file line that begins with \verb'Oval',
you will call the \verb'handleOval'
method that will then get a \verb'String' (the color)
and four \verb'double's.
The \verb'handleOval' method will then
create an \verb'OvalSprite' object with the appropriate width and height,
set its color to the given color value,
locate it at the appropriate x and y values,
and add it to the game canvas.
(Note that you got to the \verb'handleOval' method
because you already saw the token \verb'Oval'
in the \verb'readPNDFile'.)
Similar remarks apply to PND lines that begin with \verb'Rectangle'.
For \verb'Ball' lines,
you need to create a \verb'Pinball' object
(look at \verb'Pinball.java' to see what the constructor requires
for this class!),
adjust its color and location,
add it to the game canvas,
{\em and} add it to the \verb'pinballs' list.

In the PND file, the string \code{<color>} can be passed to \code{getColor}
to return a color; all the rest of the entries
for any of the game objects are double values.
The variables \code{x} and \code{y} are the
location of the bumper (or ball);
\code{w} and \code{h} are width and height.
The \code{dX}
and \code{dY} values determine the velocity of the pinball.

After you have finished the \code{readPNDFile} method
and implemented the three \verb'handle...' methods,
your program should work perfectly.
Use the \fname{first.pnd} configuration file
to test your program's behavior.

\subsection*{Checkpoint 3}
{\bf
Show us your working code at this point.
}

\section*{Write your own PND file}
Create a PND file of your own choosing
that describes a pinball level.
It's best to start with a sketch of what you want first,
so that you can get a good idea of how your objects
(bumpers and pinballs) will appear
and so you can come up with reasonable values for
sprite dimensions and locations.
Your PND file should end with a \fname{.pnd} extension.

\subsection*{Checkpoint 4}
{\bf
Run your \verb'PinballGame' file
and enter your newly created PND file as the filename.
Show us your results.
}

\section*{Are you done?}
If you have finished with the checkpoints,
start working on your Assignment 5.

\end{document}
